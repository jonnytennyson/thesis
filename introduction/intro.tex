\chapter{Introduction}
\label{chap:intro}
\section{Introduction}

\section{Fluorescence Microscopy}

\section{Forster Resonance Energy Transfer}
F\"{o}rster resonance energy transfer (FRET) is a non-radiative energy transfer process that can occur between chromophoric molecules~\cite{forster48}. The degree of energy transfer is dependent on the fluorophore distance, allowing FRET to be used as a ``molecular ruler"~\cite{stryer67}, to determine intramolecular distances. Since FRET was first used to measure the distance between two fluorescent dyes on individual molecules bound to a surface~\cite{ha96}, single-molecule FRET (smFRET) has become a popular tool to investigate the structure and dynamics of biomolecules, both on a surface and diffusing freely in solution~\cite{haran03, schuler02, weiss00}.

\section{Single Molecule Fluorescence Microscopy}
\subsection{Overview}
\subsection{Confocal Microscopy}
In a confocal smFRET experiment, biological molecules are labelled with two fluorescent dyes. The emission spectrum of the donor dye ($D$) is chosen to overlap with the excitation spectrum of the acceptor ($A$). When the donor and acceptor are sufficiently close in space, exciting the donor dye results in FRET and fluorescent emission from the acceptor dye. The FRET efficiency, $E$, which describes the proportion of excitation energy transferred from the donor to the acceptor, depends on the distance, $r$ between the two dyes (Eq.~\ref{eq:efficiency}) and $R_0$, the F\"{o}rster distance, a dye dependent constant that describes the dye separation at which 50\% energy transfer is achieved (Fig.~\ref{fig:fig1_instrumentation} C)

\begin{equation}
E = \frac{1}{1 + (\frac{r}{R_0})^6} 
\label{eq:efficiency}
\end{equation}

Consequently, the distance between the two fluorophores can be determined from the ratio of donor and acceptor photons emitted during an excitation event (Eq.~\ref{eq:Eprod}).

Experimentally, a collimated laser beam, is used to illuminate an extremely dilute solution of labelled molecules. When a labelled molecule diffuses through the laser beam, the donor dye is excited and photons are emitted from both donor and acceptor dyes.  Emitted photons are collected through the objective and separated into donor and acceptor streams for collection and analysis (Fig.~\ref{fig:fig1_instrumentation} A, B). 

For a single fluorescent burst, the FRET efficiency, $E$, can be calculated as (Eq~\ref{eq:Eprod}):

\begin{equation}
E = \frac{n_A}{n_A + \gamma \cdot n_D}
\label{eq:Eprod}
\end{equation} 

for $n_A$ and $n_D$ detected acceptor and donor photons respectively and $\gamma$ an experimentally determined instrument-dependent correction factor. During the course of a smFRET experiment, several thousand fluorescent bursts are collected and used to construct FRET efficiency histograms. These histograms can be used to identify populations of fluorescent species~\cite{ha96}, typically by fitting multiple gaussian distributions.

\subsection{Total Internal Fluorescence Miroscopy}

\subsection{Epifluorescent Microscopy}

\subsection{Super-Resolution Microscopy}

\section{Probabilistic Inference and Bayesian Statistics}
\subsection{Bayesian Statistics}
\subsection{Sampling Techniques}
