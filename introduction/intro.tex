\chapter{Introduction}
\label{chap:intro}
\section{Introduction}
This chapter provides a general introduction to the contextual background of the work presented in this thesis. First, we provide a general overview of the physical phenomena of fluorescence and F\"{o}rster resonance energy transfer (FRET). We describe the application of these phenomena to the study of biological molecules and the common experimental and analytical techniques used in these research areas. We then present an overview of techniques of statistical analysis, with a particular focus on model-based Bayesain inference. We present common sampling-based techniques for Bayesian statistical analysis and describe why Bayesian inference is a useful tool in the analysis of single-molecule fluorescence data.

\section{Fluorescence and FRET}
\subsection{The Physical Basis of Fluorescence}
\paragraph{Fluorescence}
Fluorescence is the physical phenomenon by which a molecule that has absorbed a photon of electromagnetic radiation and which is in an excited state relaxes via non-radiative processes, such as vibrational interactions, rotation and translation in its medium and then  emits a photon, returning to its ground state. Owing to the intramolecular transitions occurring prior to emission, the photon emitted is of a lower energy than the photon absorbed. The difference in energy between the absorbed and emitted photons is termed the Stokes shift (Fig.~\ref{fig:jablonski} B,~\cite{Albani2011}) and is a characteristic of the energy levels available to the excited molecule. 

The typical timescale for a single fluorescence excitation-emission cycle is approximately $10^{-10} - 10^{-7}$ s. This timescale is dominated by the lifetime of the excited state, (typically $10^{-10} - 10^{-7}$ s); photon excitation and emission processes occur on a timescale that is several orders of magniude shorter (absorption $10^{-15}$ s, relaxation $10^{-12} - 10^{-10}$ s).
 
\paragraph{Competing Processes}
The fluorescence excitation-emission cycle involves transitions between the singlet ground state ($\text{S}_0$) and first excited state ($\text{S}_1$) of the fluorophore (Fig.~\ref{fig:jablonski}). However, other competing, non-radiative processes can occur. Internal conversion -- a non-radiative transition to a lower energy state of the same spin multiplicity -- is possible, although the large energy gap between the $\text{S}_1$ and $\text{S}_0$ states makes it much less frequent than the more accessible fluorescent pathway. A more significant competitor is inter-system crossing, involving a non-radiative transition from the first singlet excited state, $\text{S}_1$ to the first triplet excited state, $\text{T}_1$. Although forbidden by spin selection rules, inter-system crossing is facilitated by spin-orbit coupling, allowing it to compete with fluorescence (rate $10^{-10} - 10^{-8}$ s) as a relaxation pathway.

From the $\text{T}_1$ state there are several accessible pathways to the ground state ($\text{S}_0$). Phosphorescence is the radiative decay from $\text{T}_1$ to $\text{T}_0$; alternatively, a second inter-system crossing back to $\text{S}_1$, followed by delayed fluorescence emission to reach the $\text{S}_0$ state is more common. As the lifetime of the $\text{T}_1$ state is long (on the order of $10^{-6} - 1$ s), this can be observed as photoblinking.  

A third relaxation pathway from the $\text{T}_1$ state is photobleaching. If the fluorophore is in a solution containing dissolved oxygen, the $\text{T}_1$ state can interact with the triplet ground state of molecular oxygen, leading to de-excitation via triplet-triplet annihilation:

\begin{equation}
^3\text{T}_1 + ^3\text{O}_2 \longrightarrow ^1\text{S}_0 + ^3\text{O}_2
\label{eq:blinking1}
\end{equation} 

The resultant singlet oxygen can then irreversibly oxidise the fluorophore, leaving it unable to undergo further fluorescent excitation-emission cycles.   

%Intersystem crossing ($10^{-10} - 10^{-8}$ s) and internal conversion ($10^{-11} - 10^{-9}$ s) are slower processes.

One useful parameter for characterising a fluorophore is its quantum yield, $\phi_f$, the fraction of excited molecules that undergo photon emission to return to the ground state. It is defined as: 

\begin{equation}
\phi_f = \frac{k_f}{k_f + k_{nr}}
\label{eq:quantum_yield}
\end{equation}

where $k_f$ and $k_{nr}$ are the rates of fluorescence emission and non-radiative emission respectively. When all molecules in the excited state were promoted by photon absorption, $\phi_f$ can be equivalently defined as the fraction of absorption events that result in fluorescence emission: 


\begin{equation}
\phi_f = \frac{n_{\text{emitted}}}{n_{\text{absorbed}}}
\label{eq:quantum_yield_ratio}
\end{equation}

where $n_{\text{emitted}}$ and $n_{\text{absorbed}}$ are respectively the number of emitted and absorbed photons.

The energetic interconversions involved in fluorescence can be summarised using a Jablonski diagram (Fig.~\ref{fig:jablonski}).

\begin{figure}[!ht]
   \begin{center}
      %\includegraphics*[clip=true, width=5in]{jablonski.pdf}
      \caption{A) Jablonski diagram depicting the possible electronic processes undergone by an excited fluorophore. Solid lines indicate electronic transitions, wavy lines indicate vibrational transitions. IC = internal conversion, ISC = intersystem crossing. B) Characteristic times of the processes depicted in the Jablonski diagram. C. Due to fast vibrational relaxation, the fluorescence emission is shifter to longer wavelengths than the absorption (Stokes shift). Figure from~\cite{Horrocks2014}, used with permission.}
      \label{fig:jablonski}
   \end{center}
\end{figure}

\subsection{The Physical Basis of FRET}
\paragraph{F\"{o}rster Resonance Energy Transfer}
F\"{o}rster resonance energy transfer (FRET) is a non-radiative energy transfer process that can occur between chromophoric molecules~\cite{forster48}. The degree of energy transfer, E, is dependent on the fluorophore distance, varying inversely with the sixth power of the dye-dye distance, $r$: 

\begin{equation}
E = \frac{1}{1+ (\frac{r}{R_0})^6}
\label{eq:FRET_d}
\end{equation}

Here, $R_0$ is the F\"{o}rster distance, a dye-dependent constant that defines the dye-dye distance for which the energetic transfer efficiency is 50 \%. $R_0$ is defined, for a donor (D) and acceptor (A) fluorophore pair, by their degree of spectral overlap and their relative orientation:

\begin{equation}
R_0 = \sqrt[\leftroot{-2}\uproot{2}6]{\frac{9000\phi_D\ln(10)\kappa^2J(\lambda)}{128\pi^2n^4N_A}}
\label{eq:R_0}
\end{equation}

Here, $\phi_D$ is the quantum yield of the donor fluorophore, $\kappa$ is the dipole orientation factor, $N_A$ is Avogadro's number, n is the refractive index of the medium, and $J(\lambda)$ is the spectral overlap integral:

\begin{equation}
J(\lambda) = \int f_D(\lambda)\epsilon_A(\lambda)\lambda^4\,d\lambda
\label{eq:J}
\end{equation}

where $f_D(\lambda)$ and $\epsilon_A(\lambda)$ are respectively the normalised emission spectrum of the donor and the molar extinction coefficient of the acceptor at wavelength $\lambda$. The dipole orientation factor, $\kappa^2$ is typically assumed to be $\frac{2}{3}$, the value observed if both dyes are freely rotating and hence can be assumed to be isotropically oriented during the lifetime of the $\text{S}_1$ excited state~\cite{Demchenko2008}.

\paragraph{Deriving the FRET Equation}
The distance dependence of the energy transfer allows FRET to be used as a ``molecular ruler"~\cite{stryer67}, to determine intramolecular distances. For a given FRET enrgy transfer event, the FRET Efficiency, $E$, defined with respect to the intramolecular distance $r$ in Eq.~\ref{eq:FRET_d}, is given ratiometrically by:

\begin{equation}
E = \frac{n_A}{\gamma \cdot n_D + n_A}
\label{eq:FRET_E}
\end{equation}

where $n_A$ and $n_D$ are the number of observed photons emitted by the acceptor and donor fluorophores respectively, and $\gamma$ is an instrument-dependent constant that corrects for unequal detection efficiencies. The equivalence of Eq.~\ref{eq:FRET_d} and Eq.~\ref{eq:FRET_E} is derived below.

As defined above, the quantum yield of a (donor) fluorophore is given by the ratio of the rates of decay from the excited state $\text{S}_1$ via radiative and non-radiative processes: 

\begin{equation}
\phi_D = \frac{k_f}{k_f + k_{nr}}
\label{eq:quantum_yield_donor}
\end{equation}

Similar ratios can be used to describe the quantum yield of a donor fluorophore that can undergo FRET energy transfer (Eq.~\ref{eq:quantum_yield_donor_FRET}) and the FRET efficiency (Eq.~\ref{eq:FRET_ratio}):

\begin{equation}
\phi_{DA} = \frac{k_f}{k_f + k_{nr} + k_{ET}}
\label{eq:quantum_yield_donor_FRET}
\end{equation}

\begin{equation}
E = \frac{k_{ET}}{k_f + k_{nr} + k_{ET}}
\label{eq:FRET_ratio}
\end{equation}

where $k_{ET}$, the rate constant for the FRET energy transfer process, is given by:

\begin{equation}
k_{ET} = \frac{1}{t_D}(\frac{R_0}{r})^6
\label{eq:k_ET}
\end{equation}

and $t_D$ is the lifetime of the excited state in the absence of an acceptor fluorophore: 

\begin{equation}
t_D = \frac{1}{k_{f} + k_{nr}}
\label{eq:ex_lifetime}
\end{equation}


\section{Single Molecule Fluorescence Microscopy}
Since FRET was first used to measure the distance between two fluorescent dyes on individual molecules bound to a surface~\cite{ha96}, single-molecule FRET (smFRET) has become a popular tool to investigate the structure and dynamics of biomolecules, both on a surface and diffusing freely in solution~\cite{haran03, schuler02, weiss00}.

\subsection{Confocal Microscopy}
\subsection{Instrumentation}
\subsection{Data Acquisition}
\subsection{Data Analysis}

\section{Forster Resonance Energy Transfer}

\section{Single Molecule Fluorescence Microscopy}
\subsection{Overview}
\subsection{Confocal Microscopy}
In a confocal smFRET experiment, biological molecules are labelled with two fluorescent dyes. The emission spectrum of the donor dye ($D$) is chosen to overlap with the excitation spectrum of the acceptor ($A$). When the donor and acceptor are sufficiently close in space, exciting the donor dye results in FRET and fluorescent emission from the acceptor dye. The FRET efficiency, $E$, which describes the proportion of excitation energy transferred from the donor to the acceptor, depends on the distance, $r$ between the two dyes (Eq.~\ref{eq:efficiency}) and $R_0$, the F\"{o}rster distance, a dye dependent constant that describes the dye separation at which 50\% energy transfer is achieved (Fig.~\ref{fig:fig1_instrumentation} C)

\begin{equation}
E = \frac{1}{1 + (\frac{r}{R_0})^6} 
\label{eq:efficiency}
\end{equation}

Consequently, the distance between the two fluorophores can be determined from the ratio of donor and acceptor photons emitted during an excitation event (Eq.~\ref{eq:Eprod}).

Experimentally, a collimated laser beam, is used to illuminate an extremely dilute solution of labelled molecules. When a labelled molecule diffuses through the laser beam, the donor dye is excited and photons are emitted from both donor and acceptor dyes.  Emitted photons are collected through the objective and separated into donor and acceptor streams for collection and analysis (Fig.~\ref{fig:fig1_instrumentation} A, B). 

For a single fluorescent burst, the FRET efficiency, $E$, can be calculated as (Eq~\ref{eq:Eprod}):

\begin{equation}
E = \frac{n_A}{n_A + \gamma \cdot n_D}
\label{eq:Eprod}
\end{equation} 

for $n_A$ and $n_D$ detected acceptor and donor photons respectively and $\gamma$ an experimentally determined instrument-dependent correction factor. During the course of a smFRET experiment, several thousand fluorescent bursts are collected and used to construct FRET efficiency histograms. These histograms can be used to identify populations of fluorescent species~\cite{ha96}, typically by fitting multiple gaussian distributions.

\subsection{Total Internal Fluorescence Miroscopy}

\subsection{Epifluorescent Microscopy}

\subsection{Super-Resolution Microscopy}

\section{Probabilistic Inference and Bayesian Statistics}
\subsection{Bayesian Statistics}
\subsection{Sampling Techniques}
