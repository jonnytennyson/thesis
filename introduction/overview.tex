\chapter*{Summary}
%\section{General Introduction}
%\section{Contributions}
%\section{Thesis Overview}
Single-molecule fluorescence microscopy describes a number of experimental techniques for the study of individual molecules using fluorescence detection. This experimental technique has found a wide range of research applications. These applications include the study of protein folding, protein aggregation and intermolecular associations. Since the first demonstration of the detection of a single, fluorescently-labelled molecule, considerable effort has gone into to improving the quality of the data obtained, through the development of novel experimental methodologies.

However, the development of novel experimental methods requires the concomitant development and rigorous evaluation of analysis tools that can maximise the information available from the data obtained. This thesis describes the development and thorough evaluation of data analysis methodologies for confocal single-molecule fluorescence microscopy. Firstly, the thesis covers the implementation of an open source software library for the analysis of confocal single-molecule fluorescence data, and the systematic evaluation of existing data analysis methodologies. Secondly, a novel method of data analysis, based on Monte Carlo sampling, is introduced. This analysis methodology is applied to the calculation of intramolecular distances and to the determination of oligomer sizes using confocal fluorescence microscopy. A thorough evaluation of the performance of these analysis methods is performed. A final chapter describes the development of an error correction tool for genome assemblies generated using fluorescence-based sequencing technology from Illumina.

Overall, the work reported in this thesis describes the development and systematic evaluation of computational techniques for the analysis of fluorescence data. The thesis describes the theoretical basis of novel techniques and the deployment and analysis of both novel and existing methodologies, as well as a thorough experimental validation of the analysis methods developed. In completing this work, I hope to provide a stable foundation on which further research can be built, both through application of the techniques described here and through their extension to further experimental techniques.

%To date, the same attention has not been paid to data analysis. Analysis of single-molecule fluorescence data is typically considered of secondary importance to data collection. As a consequece of this, the development and evaluation of data analysis tools for single-molecule fluorescence microscopy have lagged behind development of experimental techniques. 



%The rest of this thesis is structured as follows. Chapter 2 provides a introduction to other research that has been undertaken in the field of fluorescence microscopy. General experimental and analysis techniques for fluorescence microscopy of single molecules are introduced, and the work presented in this thesis is contextualised. Chapter 2 also provides an overview of current research in Bayesian statistics and probabilistic analysis, introducing the statistical methods that are used in later chapters. Following this introductory section, we present our results over four separate chapters.

%Chapter 3 introduces the pyFRET library, which we developed for analysis of confocal smFRET data. We describe the theory and implementation of different analysis algorithms for smFRET datasets. Data from both continuous and alternating excitation experiments are considered. In the second part of Chapter 3, we provide a comprehensive evaluation of different smFRET analysis algorithms, using a combination of simulated and experimental datasets. We benchmark popular analysis techniques, demonstrating their relative utility under different data collection regimes.

%Chapter 4 considers a Bayesian method for the analysis of data from continuous excitation smFRET datasets. First, we introduce a model-based theory of the smFRET experiment. Then, we describe how this parametric model can be used to infer intramolecular distances and population sizes from time-binned smFRET data. We benchmark our Bayesian analysis technique against thresholding techniques used for time-binned data, showing the superior performance of the inference technique.

%Chapter 5 extends this Bayesian analysis to sizing of labelled protein aggregates. We describe how a simplified model can be used to describe photon emission from multiple fluorophores. Using a combination of real and simulated datasets, we then show that this model is degenerate, making inference of aggregate sizes undecidable. We further show that a single emission event has multiple sources of heterogeneity, creating a complex, non-linear relationship between aggregate size and the number of photons emitted in a fluorescent event.

%Chapter 6, the final results chapter, is somewhat different. This chapter describes a Bayesian analysis tool for error correction in genome assemblies generated using Illumina Nextera mate pairs. Illumina sequencing technology uses fluorescence emission from multiple fluorophores in its base calling algorithm. However, base calling errors and complex repeat structures can make reassembly of short reads challenging. In this chapter, we introduce NxRepair, an error correction tool that can identify mistakes in {\emph de novo} assemblies of bacterial genomes. We benchmark NxRepair against existing tools, demonstrating its superior performance.

%Finally, Chapter 7 provides the conclusion to the thesis. Here, we summarise the overall contribution of this thesis and relate the work described to its wider research context. We also discuss possible extensions of the research described, indicating future applications of the research.