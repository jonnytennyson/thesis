\chapter{Analysis Tools for single-molecule Confocal Microscopy}
\label{chap:pyfret}
\section{Overview}
This chapter describes the development of pyFRET, an open source library of analysis tools for confocal single-molecule spectroscopy. The chapter is structured as follows. Firstly, the analysis algorithms supported by pyFRET, including their theoretical basis and programmatic implementation, are described. Next the performance of different analysis techniques, as implemented using pyFRET, are compared to understand their strengths. 

The contributions of this chapter are twofold. Firstly, pyFRET is the first open source software released for smFRET data analysis. This is important as open software facilitates effective benchmarking and comparison of different analysis methods. Secondly, the first detailed comparison of different analysis techniques for confocal smFRET is presented. The data collection and analysis methods used by different research groups are currently highly heterogeneous, so this comparison makes an important contribution to understanding of the best available techniques to use for accurate evaluation of smFRET data.  

\section{Introduction}
\subsection{The single-molecule Fluorescence Experiment}

As described in Chapter~\ref{chap:intro}, in a confocal smFRET experiment, molecules are labelled with two fluorescent dyes. The emission spectrum of the donor dye ($D$) is chosen to overlap with the excitation spectrum of the acceptor ($A$). When the donor and acceptor are sufficiently close in space, exciting the donor dye results in FRET and fluorescent emission from the acceptor dye.  The FRET efficiency, $E$, which describes the proportion of excitation energy transferred from the donor to the acceptor, depends on the distance, $r$ between the two dyes (Eq.~\ref{eq:efficiency}) and $R_0$, the F\"{o}rster distance, a dye dependent constant that describes the dye separation at which 50\% energy transfer is achieved (Fig.~\ref{fig:fig1_instrumentation} C)

\begin{equation}
E = \frac{1}{1 + (\frac{r}{R_0})^6} 
\label{eq:efficiency}
\end{equation}

Consequently, the distance between the two fluorophores can be determined from the ratio of donor and acceptor photons emitted during an excitation event (Eq.~\ref{eq:Eprod}). Some typical values of $R_0$ for commonly used FRET pairs of fluorophores are given in Table~\ref{tab:R0_examples}.


\begin{table}[!ht]
\caption{Example $R_0$ values for some Alexa Fluor FRET pairs. These values, given in Angstroms, are provided by the manufacturer~\cite{sigma_alexa}.}
\begin{tabular}{|c|ccc|}
\hline
 & & Acceptor & \\
\hline
Donor & Alexa Fluor 568 & Alexa Fluor 594 & Alexa Fluor 647 \\
\hline
488 & 62 & 60 & 56 \\
546 & 70 & 71 & 74 \\
555 & & 47 & 51 \\
\hline
\end{tabular}
\label{tab:R0_examples}
\end{table}

\begin{figure}[!ht]
   \begin{center}
      \includegraphics*[clip=true, width=6in]{pyFRET/Fig1_schematic.pdf}
      \caption{Instrumentation for a smFRET experiment. A) The confocal microscope, excitation and detection apparatus. B) Labelled molecules diffuse through the excitation volume. C) The characteristic sigmoidal dependence of FRET efficiency on dye-dye distance.}
      \label{fig:fig1_instrumentation}
   \end{center}
\end{figure}


Experimentally, a focused laser beam, is used to illuminate an extremely dilute solution of labelled molecules. When a labelled molecule diffuses through the laser beam, the donor dye is excited and photons are emitted from both donor and acceptor dyes.  Emitted photons are collected through the objective and separated into donor and acceptor streams for collection and analysis (Fig.~\ref{fig:fig1_instrumentation} A, B). 

For a single fluorescent burst, the FRET efficiency, $E$, can be calculated as (Eq~\ref{eq:Eprod}):

\begin{equation}
E = \frac{n_A}{n_A + \gamma \cdot n_D}
\label{eq:Eprod}
\end{equation} 

for $n_A$ and $n_D$ detected acceptor and donor photons respectively and $\gamma$ an experimentally determined instrument-dependent correction factor. This correction factor accounts for differences in detection efficiency of donor and acceptor photons caused by the specific long- and band-pass filters used in the instrument setup, as well as the effect of the excitation powers on photon emision from the two fluorophores. During the course of a smFRET experiment, several thousand fluorescent bursts are collected and used to construct FRET efficiency histograms, which can then be used to identify populations of fluorescent species~\cite{ha96}.

\section{Data Analysis in Confocal smFRET Experiments}
Analysis of smFRET data involves several computational challenges. Firstly, photons emitted by a fluorescent molecule diffusing through the excitation volume must be identified against a noisy background. Secondly, identified bursts must be denoised, including removal of background auto-fluorescence and donor-acceptor crosstalk. Fluorescent bursts that are distorted by photobleaching or other photophysical artifacts should be identified and excluded. Multiple methods of burst selection and analysis have been developed and applied to the analysis of smFRET data~\cite{weiss00, deniz01, gell06, nir06, kapanidis05, muller05, doose07, kudryavtsev2012, eggeling01}. However, software for analysis of smFRET data has thus far been developed on an ad hoc basis, with individual groups preparing and maintaining their own analysis scripts, leading to problems typical of research programming projects~\cite{wilson06, merali10} (see Appendix~\ref{app:implementation}).


\subsection{Continuous Excitation}
\paragraph{Time-binned Data.}
\label{par:time-binned_data}
In the most simple smFRET experiment, fluorescently labelled molecules are excited by a laser that will excite the donor dye; all photons reaching the detectors during data acquisition binned as they are received into time-bins of length similar to the expected dwell-time of a molecule in the confocal volume (for small, freely diffusing molecules, a bin-time of 1 ms is typically used). Event selection then simply involves identifying time-bins that contain sufficient photons to meet a specified criterion. Two thresholding criteria are in common use. AND thresholding selects time bins for which $n_D > T_D$ AND $n_A > T_A$ for $n_D$ and $n_A$ photons in the donor and acceptor channels respectively, and $T_D$ and $T_A$ the donor and acceptor thresholds. In a similar manner, SUM thresholding considers the sum of photons observed in the donor and acceptor channels, selecting time bins for which $n_D + n_A > T$.

\paragraph{Fluorescent Burst Data.}
\label{par:burst_data}
Although using time-bins that are matched to the dwell-time of molecules in the confocal volume is simple, it is not ideal, as some bursts will be split over several bins, so may be counted as separate events, or not considered for analysis. More sophisticated event selection algorithms, typically called burst search algorithms~\cite{nir06}, bin photons on a time-scale much shorter than the typical dwell time in the confocal volume and then scan the resultant photon stream for bursts of a specified duration and brightness. pyFRET implements both All Photons Burst Search (APBS) and a Dual Channel Burst Search (DCBS) algorithms both for ALEX data and for simple FRET data, as originally described~\cite{nir06}.

In APBS burst search for FRET data, photons from both donor and acceptor channels are considered together. A burst is defined according to three constants: $T$, the averaging window; $M$, the minimum number of photons within window $T$; and $L$, the minimum total number of photons required for an identified burst to be retained. These three values are used in a two-step process for burst identification.

Firstly, ``the start (respectively, the end) of a potential burst is detected when the number of photons in the averaging window of duration T is larger (respectively, smaller) than the minimum number of photons M."~\cite{nir06}.

The DCBS burst search is similar, but considers the donor and acceptor channels separately. For a burst to be accepted in DCBS, both channels must simultaneously meet the running sum criterion, allowing exclusion of single colour bursts and bursts where one fluorophore bleaches.

\subsection{Alternating Laser Excitation}
\paragraph{Time-binned Data.}
A more sophisticated smFRET experiment uses Alternating Laser Excitation (ALEX) during data acquisition. In this method, the diffusing fluorescent molecules are subjected to excitation from two lasers in rapid alternation~\cite{kapanidis05}. One laser excites the donor fluorophore and the other can directly excite the acceptor fluorophore. The alternation of the laser excitation is fast on the timescale of molecular dwell-time in the confocal volume, allowing a single fluorescent molecule to receive multiple cycles of donor-acceptor direct excitation.

Instead of the two photon streams -- donor and acceptor photons -- observed in a simple smFRET experiment, in a confocal ALEX experiment, there are four streams: $F_{D_{ex}}^{D_{em}}$, $F_{D_{ex}}^{A_{em}}$, $F_{A_{ex}}^{D_{em}}$ and $F_{A_{ex}}^{A_{em}}$. $F_{D_{ex}}^{D_{em}}$ and $F_{D_{ex}}^{A_{em}}$, respectively donor and acceptor emission during donor excitation, are analogous to the two original donor and acceptor photon streams in a smFRET experiment. $F_{A_{ex}}^{A_{em}}$ records acceptor emission during direct acceptor excitation, whilst $F_{A_{ex}}^{D_{em}}$ records donor emission during acceptor excitation. Initial event selection is performed based on the number of photons observed during both donor excitation and acceptor excitation: $F_{D_{ex}}^{D_{em}} + F_{D_{ex}}^{A_{em}} > T_{D_{ex}}$ AND $F_{A_{ex}}^{A_{em}} > T_{A_{ex}}$ for threshold $T_{D_{ex}}$ during donor excitation and $T_{A_{ex}}$ during acceptor excitation provides an equivalent to AND thresholding; $F_{D_{ex}}^{D_{em}} + F_{D_{ex}}^{A_{em}} + F_{A_{ex}}^{A_{em}} > T$ for overall threshold $T$ is analogous to SUM thresholding. Performing event selection in this manner, based on direct excitation of both fluorophores, should remove the biases caused by simple AND or SUM thresholding. The presence of these extra channels also provides additional information about the labelling state of molecules giving rise to fluorescent bursts, allowing calculation of an emission stoichiometry, $S$:

\begin{equation}
S = \frac{F_{D_{ex}}^{D_{em}} + F_{D_{ex}}^{A_{em}}}{F_{D_{ex}}^{D_{em}} + F_{D_{ex}}^{A_{em}} + F_{A_{ex}}^{A_{em}}}
\label{eq:stoichiometry}
\end{equation}

Values of $S$  that are very close to one or very close to zero, indicate presence of only the donor or acceptor fluorophore respectively, so can be excluded from further analysis using a second event selection criterion: $S_{min} < S < S_{max}$. 

\paragraph{Fluorescent Burst Data}
The burst search algorithms implemented for ALEX data work in a similar manner to those implemented for simple FRET data. In the ALEX APBS method, bursts are identified by considering the total number of fluorescent photons $F_{total} = F_{D_{ex}}^{D_{em}} + F_{D_{ex}}^{A_{em}} + F_{A_{ex}}^{A_{em}}$ in each time bin. Bursts are identified where the running sum (calculated using the  $F_{total}$ photon stream) in the averaging window $T$ exceeds the threshold $M$. The DCBS method considers donor excitation photons $F_{D_{ex}}^{D_{em}} + F_{D_{ex}}^{A_{em}}$ separately from photons emitted during direct acceptor excitation $F_{donor} = F_{A_{ex}}^{A_{em}}$, requiring that the running sum exceeds $M$ for both $F_{donor}$ and $F_{A_{ex}}^{A_{em}}$.

paragraph{RASP: Recurrence Analysis of Single Particles}
A recent innovation in confocal smFRET is Recurrence Analysis of Single Particles (RASP)~\cite{hoffmann11}. RASP exploits the fact that fluorescent bursts occurring close in time are more likely to be from the same molecule diffusing back through the confocal volume than from a newly observed molecule. This can be used to determine interconversion kinetics and to test whether broad peaks in the FRET histogram derive from overlapping static populations.

RASP is a two-step process. First, initial bursts ($b_1$) with a FRET efficiency $E_{b1}$ within some defined range $\Delta(E_{b1})$ are identified. Secondly, bursts ($b_2$) occurring within a time interval (called the recurrence interval) $T = (t_1, t_2)$ of $b_1$ are identified. Analysis of the distribution of FRET efficiencies in $b_2$, the population of recurrent bursts, provides information about the interconversion rate between subpopulations. The rate constants of interconversion can be extracted by fitting the relative subpopulations as a function of the recurrence interval $T$.

pyFRET implements RASP using array masking, to allow efficient selection of relevant bursts. RASP can be called in a single step from a FRET bursts or ALEX bursts object, and a loop can readily be made to repeat the process at different time intervals:

\begin{lstlisting}
# RASP

# initial E range: 0.4 < E < 0.6
Emin = 0.4
Emax = 0.6

# Time interval for re-occurrence
# given in number of bins
Tmin = 1000
Tmax = 10000

# selecting re-occurring bursts
recurrent_bursts = bursts_APBS.RASP(Emin, Emax, Tmin, Tmax)

# histogram of re-occurring bursts
recurrent_bursts.build_histogram(filepath, csvname, gamma=g_factor)
\end{lstlisting} 


\section{Experimental Methods}
In order to effectively evaluate the performance of the pyFRET library, simulated datasets with known parameters were generated. smFRET data were also collected under various different excitation and data collection regimes. These datasets were then analysed using different algorithms reported in the literature and reimplemented in pyFRET. The results were compared, to determine the most effective analysis methods for smFRET data from both simple FRET and ALEX experiments. The following section describes the experimental protocols used in the generation of simulated datasets and the methods used in the collection and analysis of smFRET data.

\subsection{Instrumentation}
\label{sect:instrumentation}
Data were collected by the author using a custom-built single-molecule confocal microscope, which was constructed to perform both FRET and ALEX data collection. Briefly, two expanded, focused laser beams (488 nm (blue) and 640 nm (red) wavelengths) were directed using a fibre-optic cable (the iFlex Viper system) into the back port of an inverted microscope. The overlapped beams were focused through an oil immersion objective (Nikon CFI Plan Apochromat VC 100X Oil N2 NA 1.4, W.D 0.13 mm) into the sample. Emitted photons were collected through the same objective and separated from the excitation light using a dichroic mirror (FF500/646-Di01 (Semrock)). Out-of-focus fluorescence was removed by focusing the photons through a 50 $\mu$M pinhole (Thorlabs) using the microscope tube lens. Retained photons were further separated into donor and acceptor wavelengths by a second dichroic mirror (585DRLP (Horiba)). The longer wavelength photons were focused by a lens (Plano apo convex, focal length = 50 mm, Thorlabs) through a long-pass (565ALP (Horiba)) and band-pass (695AF55 (Horiba)) filter onto the APD detector. The shorter wavelength photons were reflected by the dichroic and focused through a long-pass (540LP (Omega)) and band-pass (535AF55 (Horiba)) filter set onto the second APD. Outputs from the two APDs are connected to a custom-programmed field-programmable gate array (FPGA) (Colexica), which counts the signals and combines them into bins of user-specificed length. 

For FRET data collection only the 488 nm laser was used. For ALEX data collection both lasers were used and the FPGA was also used to rapidly modulate laser emission via TTL pulses and separate the photons detected into donor-excitation and acceptor-excitation subsets within each time bin.  

\subsection{Benchmarking the Gaussian Fitting Using Simulated Datasets}
\label{subsec:benchmarking-gaussian}
To test the ability of the fitting algorithms to distinguish fluorescent populations with similar FRET efficiencies,  simulated datasets were generated, consisting of mixtures of FRET bursts with a known FRET efficiency and population size. These bursts were then fitted using pyFRET's two component mixture model and the results compared to the known input FRET efficiencies and population sizes. The python script used to generate and fit these datasets can be found in Appendix~\ref{app:code_samples_GMM}.

\subsection{Data to Evaluate the Simple Event Selection Algorithms}
\label{subsec:event-selection-data}

The pyFRET library was tested using DNA duplexes dual-labelled with Alexa Fluor 488 and Alexa Fluor 647. The duplex sequences and labelling sites are shown in Tables~\ref{tab:donor} and~\ref{tab:acceptors}. Labelled duplexes were diluted to a concentration of 50 pM in TEN buffer (10 mM Tris, 1mM EDTA, 100 mM Nacl), pH 8.0, containing 0.0001 \% Tween-20. FRET data were collected for 15 minutes using continuous excitation at 488 nm at a power of 80 mW. Collected photons were binned online in intervals of 1 ms and stored in files of 10000 bins. To process the data, AND thresholding was performed using thresholds $N_D = 10$ and $N_A = 10$. SUM thresholding used a threshold $N = 20$. For FRET efficiency calculation the $\gamma$-factor was 0.95.

ALEX data were collected for 15 minutes using alternating excitation at 488 and 640 nm, with respective laser powers of 80 and 70 mW, and a modulation rate of 0.1 ms, a dead-time of 0.1 $\mu$s and a delay compensation of 3 $\mu$s. ALEX data were then binned in intervals of 1 ms.

\begin{table}[!ht]
\caption{Donor DNA Sequence DNA sequence of the donor-labelled strand, where {\bf 5} is a deoxy-T nucleotide, labelled with Alexa Fluor 488 at the C6 amino position.}
\begin{tabular}{|l|l|}
\hline
Construct & Sequence \\
\hline
Donor & \footnotesize{TACTGCCTTTCTGTATCGC{\bf 5}TATCGCGTAGTTACCTGCCTTGCATAGCCACTCATAGCCT} \\
\hline
\end{tabular}
\label{tab:donor_1}
\end{table}

\begin{table}[!ht]
\caption{
{Acceptor DNA Sequences}. Preparing the dual-labelled dsDNA. An acceptor-labelled ssDNA, with the sequence shown was annealed to the indicated donor construct, to yield a dual-labelled construct with the labels separated by the given number of base pairs. In the displayed acceptor-strand sequences, {\bf 6} is a deoxy-T nucleotide, labelled with Alexa Fluor 647 at the C6 amino position.
}
\begin{tabular}{|l|l|}
\hline
Separation / bp & Acceptor Sequence \\
\hline
4 & \footnotesize{AGGCTATGAGTGGCTATGCAAGGCAGGTAACTACGCGATAAGCGA\bf{6}} \\
6 & \footnotesize{AGGCTATGAGTGGCTATGCAAGGCAGGTAACTACGCGATAAGCGATA\bf{6}} \\
8 & \footnotesize{AGGCTATGAGTGGCTATGCAAGGCAGGTAACTACGCGATAAGCGATACA\bf{6}} \\
10 & \footnotesize{AGGCTATGAGTGGCTATGCAAGGCAGGTAACTACGCGATAAGCGATACAGA\bf{6}} \\
12 & \footnotesize{AGGCTATGAGTGGCTATGCAAGGCAGGTAACTACGCGATAAGCGATACAGAAA\bf{6}} \\
\hline
\end{tabular}
\label{tab:acceptors}
\end{table}

\subsection{Data to Evaluate Event Selection Using the Burst Search Algorithms}
To test the burst search algorithms, data were collected under both ALEX and FRET conditions. Data were collected for 10 minutes, using laser powers of 130 $\mu$W in both donor and acceptor excitation. For FRET data collection, laser illumination by the donor laser was continuous and the data were binned online into time bins of 50 $\mu$s, roughly 5 \% of the duration of the average burst from a freely diffusing molecule. The acceptor laser was not used. For ALEX data collection, photons were similarly binned online into time bins of 50 $\mu$s length, but the laser modulation rate was increased to 50 $\mu$s, with a dead-time of 0.1 $\mu$s and a delay compensation of 3 $\mu$s, corresponding to 2 modulations per short time bin. 

To evaluate the effect of bin-time on performance of the burst search algorithms, a further dataset was collected using FRET excitation on the 6 bp duplex. For this dataset, data were collected for 10 minutes using the 488 nm laser at 130 $\mu$W. Data were binned into time bins of 10 $\mu$s, allowing for re-binning into longer time-bins as required.

Unless otherwise specified, DCBS was carried out using parameters $T = 20$ bins, $L = 10$ and $M = 10$; APBS was carried out using parameters $T = 20$ bins, $L = 20$ and $M = 20$. Scripts that reproduce these analyses can be found in Appendix~\ref{app:burst_search}.

\subsection{Performance Analysis Using Mixtures of DNA Duplexes}
To test the accuracy of the burst search algorithms, ALEX data were collected from mixtures of two DNA duplexes at known concentration ratios. The 4, 8 and 12 bp duplexes were used (see Table~\ref{tab:acceptors}). Samples were prepared containing a mixture of two duplexes in TEN buffer with a total DNA concentration of 80 pM, divided in either a 3 : 1 ratio (high concentration 60 pM, low concentration 20 pM) or a 1 : 1 ratio (both components 40 pM). Data were collected under ALEX excitation for 10 minutes, using 50 $\mu$s time-bins as described above. The data were later re-binned into 1 ms time-bins, to compare performance of burst search and simple thresholding in separating the two populations. Scripts that reproduce these analyses can be found in Appendix~\ref{app:code_samples_ALEX_mix} and Appendix~\ref{app:code_samples_FRET_mix}. All fits were performed using the Gaussian Mixture Model (GMM) implemented in Scikit-learn~\cite{scikit-learn}, which uses the Expectation-Maximisation algorithm to perform the fit.  


\subsection{Testing the RASP Algorithm}
To test the RASP Algorithm, a 1:1 mixture of 6 bp and 12 bp duplex was prepared to a total DNA concentration of 50 pM. A 400 $\mu$L aliquot of the dilute solution was placed in one chamber of a lidded, chambered coverslide (LabTex) to reduce evaporation during the measurement. FRET data were collected for 600 minutes using continuous excitation at 488 nm at a power of 140 $\mu$W. Collected photons were binned online in intervals of 50 $\mu$s and stored in files of 1000000 bins. Events were identified using APBS burst selection was used, with the lower thresholds of $L = 10$ and $M = 10$ to maximise the number of detected events. RASP was performed on the selected events, using the parameters given in Table~\ref{tab:RASP}. 

\begin{table}[!ht]
\caption{
{RASP Parameters}. Parameters used in RASP analysis of the 6 bp duplex - 12 bp duplex mixture.}
\begin{tabular}{|l|l|}
\hline
Parameter & Value \\
\hline
T & 20 \\
M & 10 \\
L & 10 \\
$E_{min}$ & 0.65 \\
$E_{max}$ & 0.85 \\
$\Delta$T & 1 ms \\
\hline
\end{tabular}

\label{tab:RASP}
\end{table}

\section{Performance Analysis of smFRET Analysis Algorithms}
\label{sect:performance_analysis}
To evaluate the performance of the smFRET analysis algorithms, experimental datasets collected as described above were used. The following sections describe the data analysis pathway using pyFRET and present the results of pyFRET analysis. A comparison of the performance of different algorithms on smFRET data collected from the same DNA constructs but under different excitation and photon collection regimes is also presented.

\subsection{Evaluating Performance with DNA Duplexes}
\paragraph{Simple Algorithms.}
The pyFRET library was tested using DNA duplexes dual-labelled with Alexa Fluor 488 and Alexa Fluor 647. The duplex sequences, dye attachment sites and dye-dye separations are shown in Tables~\ref{tab:donor_1} and~\ref{tab:acceptors}. Event selection and denoising, calculation of FRET efficiency and the plotting and fitting of FRET efficiency histograms were performed using pyFRET. The results, shown in Fig.~\ref{fig:fig4_AND_plots} for FRET data analysed using AND thresholding, and Fig.~\ref{fig:fig5_ALEX_plots} for ALEX data, demonstrate that even using the simplest event selection and denoising techniques, pyFRET is able to effectively fit histograms from single fluorescent populations (Fig.~\ref{fig:fig4_AND_plots} and Fig.~\ref{fig:fig5_ALEX_plots} A - E), to reproduce the characteristic sigmoidal FRET efficiency curve (Fig.~\ref{fig:fig4_AND_plots} and Fig.~\ref{fig:fig5_ALEX_plots} F). However, as can be seen from Fig.~\ref{fig:fig6_Eplots} C), where the FRET efficiency curves from four different analysis algorithms are overlaid, the simple FRET analysis results in a flattened curve, caused by the bias towards events with intermediate FRET efficiencies displayed by the simple AND thresholding algorithm. 

\begin{figure}[!ht]
   \begin{center}
      \includegraphics*[clip=true, width=6.5in]{pyFRET/FRET_AND.pdf}
      \caption{Analysis of FRET data from DNA duplexes using pyFRET. A - E: Fitted FRET histograms from DNA duplexes labelled with a dye-dye separation of 4, 6, 8, 10 and 12 base pairs respectively. These fits were performed using the Gaussian Mixture Model (GMM) implemented in Scikit-learn~\cite{scikit-learn}, which uses the Expectation-Maximisation algorithm to perform the fit. F) Characteristic sigmoidal curve of FRET efficiency against dye-dye distance.}
      \label{fig:fig4_AND_plots}
   \end{center}
\end{figure}

\begin{figure}[!ht]
   \begin{center}
      \includegraphics*[clip=true, width=6.5in]{pyFRET/FRET_ALEX.pdf}
      \caption{Analysis of ALEX data from DNA duplexes using pyALEX. A - E: Fitted FRET histograms from DNA duplexes labelled with a dye-dye separation of 4, 6, 8, 10 and 12 base pairs respectively. These fits were performed using the Gaussian Mixture Model (GMM) implemented in Scikit-learn~\cite{scikit-learn}, which uses the Expectation-Maximisation algorithm to perform the fit. All data displayed, including the tails were included in the fit. F) Characteristic sigmoidal curve of FRET efficiency against dye-dye distance. }
      \label{fig:fig5_ALEX_plots}
   \end{center}
\end{figure}


\paragraph{Burst Search Algorithms}
The pyFRET burst search algorithms were tested using data collected from the same DNA duplexes but using a shorter bin-time. Sample results, from DCBS analysis of both FRET and ALEX data, and APBS analysis of ALEX data are shown in Fig.~\ref{fig:burst_search}. APBS analysis of the FRET burst data is not shown, as the presence of a zero peak hampered fitting of the low-FRET species. The characteristic sigmoidal FRET efficiency curves generated from DCBS analysis of all five duplexes are shown in Fig.~\ref{fig:fig6_Eplots} A (FRET data) and B (ALEX data).

\begin{landscape}
\begin{figure}[!ht]
   \begin{center}
      \includegraphics*[clip=true, width=9in]{pyFRET/burst_search.pdf}
      \caption{Testing the burst search algorithms. A) Fitted FRET histograms from DCBS on FRET data. B) Fitted FRET histograms from APBS on ALEX data. C) Fitted FRET histograms from DCBS on ALEX data. In A-C histograms from left to right are collected from 4 bp, 6 bp, 8 bp, 10 bp and 12 bp duplexes respectively. Histograms from APBS on FRET data are not shown, as the large zero peak hampers Gaussian fitting when the FRET efficiency is low. This is because, at low FRET efficiencies there is a significant overlap between the position of the zero peak and the position of the FRET peak. When the zero-peak is large, this can entirely mask the true FRET distribution.}
      \label{fig:burst_search}
   \end{center}
\end{figure}
\end{landscape}

\begin{figure}[!ht]
   \begin{center}
      \includegraphics*[clip=true, width=3in]{pyFRET/Bp_vs_E.pdf}
      \caption{Plot of FRET Efficiency vs dye-dye separation. A) FRET data, analysed using the DCBS burst search algorithm. B) ALEX data, analysed using the DCBS burst search algorithm. C) Comparison of different methods. Blue circles and red triangles show ALEX and FRET burst data respectively; green crosses and open circles show simple thresholded data from ALEX and FRET experiments respectively.}
      \label{fig:fig6_Eplots}
   \end{center}
\end{figure}

\subsection{Evaluating the Burst Search Algorithms}
In addition to demonstrating the functionality of the burst search algorithms, a comprehensive analysis of the impact of bin-time, threshold and detection window on the performance of the burst search algorithms is described. To our knowledge, this is the first time that such an analysis has been performed.

Burst search algorithms were developed as an improvement to the simple thresholding technique, designed to reduce inaccuracies caused by photobleaching and by long bursts being split over multiple time-bins. To assess the improvement in data quality as a result of using the burst search algorithm, data were collected from the 6 bp duplex using short time-bins of 10 $\mu$s, which could then be re-binned into longer time-bins for comparative analysis. Analysis was performed using both APBS and DCBS burst search algorithms on FRET data. For APBS, thresholds of $M=20$ and $L=20$ were used; the thresholds used for DCBS were $M = 10$ and $L = 10$. As the bin-time was varied, the minimum burst duration $T$, given in number of bins, was also varied, to keep the minimum burst duration to a constant time of 1 ms. The bin lengths and their corresponding value of T are shown in Table~\ref{tab:bin-times}. The results, shown in Fig.~\ref{fig:fig7_binning} A (DCBS) and B (APBS) are surprising. When the minimum time-interval for burst detection is not varied, the performance of the algorithm is essentially unaffected by the bin-times used: for both APBS and DCBS algorithms, the resultant FRET efficiency histograms are extremely similar, whether many short bins or a few long bins are searched.


\begin{table}[!ht]
\caption{Bin-times. Bin lengths and the corresponding minimum burst duration used to evaluate the effect of bin-time on burst search performance.}
\begin{tabular}{|l|l|}
\hline
Bin-time / $\mu$s & T / bins \\
\hline
10 & 100 \\
20 & 50 \\
40 & 25 \\
50 & 20 \\
100 & 10 \\
500 & 2 \\
1000 & 1\\
\hline
\end{tabular}

\label{tab:bin-times}
\end{table}

Secondly, the effect of the search window T on burst search performance is evaluated. For a fixed bin-time of 50 $\mu$s,  the required burst duration T was varied between 100 $\mu$s (2 bins) and 1000 $\mu$s (10 bins). The results, shown in Fig.~\ref{fig:fig7_binning} C for the DCBS algorithm are again surprising. Across all values tested, the shape of the FRET efficiency histogram is unaffected by the size of the burst search window. The peak areas are also relatively unaffected by the search window size: the very shortest window of 100 $\mu$s retains only the brightest bursts, resulting in a slightly reduced peak area; other than this the number of detected events is not significantly altered. An analysis of DCBS and APBS on ALEX data showed similar results (Fig.~\ref{fig:fig7_binning} D - F), although the reduction in peak area when the time window is shortest (100 $\mu$s) is more pronounced (Fig.~\ref{fig:fig7_binning} F). This lack of dependence on the burst search parameters, observed for both ALEX and FRET data, occurs because the bright burst are recognisable both when many short bins or few long bins are considered, ensuring that conclusions drawn are independent of the thresholding parameters used.

\begin{figure}[!ht]
   \begin{center}
      \includegraphics*[clip=true, width=6.5in]{pyFRET/window_effect.pdf}
      \caption{Evaluating the effect of detection window and bin time on the DCBS burst search algorithm for FRET and ALEX data. A - C show analysis of FRET data. D - F show analysis of ALEX data. A) Varying the length of the time-bin has little effect on the performance of the DCBS algorithm. B) Varying the length of the time-bin has little effect on the performance of the APBS algorithm. C) Reducing the length of the minimum detection window used in DCBS selects for very bright bursts. For very short windows, this reduces the number of detected bursts but does not affect the calculated FRET efficiency. D) Varying the length of the time-bin has little effect on the performance of the DCBS algorithm. E) Varying the length of the time-bin has little effect on the performance of the APBS algorithm. D) Reducing the length of the minimum detection window used in DCBS selects for very bright bursts.}
      \label{fig:fig7_binning}
   \end{center}
\end{figure}

Finally, the effect of the thresholds M and L (described in Section~\ref{par:burst_data}) on the performance of the DCBS algorithm was evaluated for both FRET and ALEX data using the high-FRET 6 bp duplex.  The thresholds used in burst search analysis were systematically varied, then evaluated their effect on the fitted peak area and FRET efficiency. The results, shown in Fig.~\ref{fig:fig8_heatmaps}, display several interesting features. Firstly, the decline in peak area with increased threshold is striking for both FRET (Fig.~\ref{fig:fig8_heatmaps} C) and ALEX (Fig.~\ref{fig:fig8_heatmaps} D) data, suggesting that the lowest possible values of L and M should be used, to maximise the data retained. Secondly, increasing the thresholds systematically reducuces the calculated FRET efficiency for the FRET dataset (Fig.~\ref{fig:fig8_heatmaps} A). This is becausethe DCBS algorithm is biased against bursts that have a low donor count, as they do not meet the initial threshold M. This effect is not seen in DCBS on ALEX data (Fig.~\ref{fig:fig8_heatmaps} B), as events are selected based on photons emitted during direct donor and acceptor excitation, so there is no bias towards intermediate FRET efficiencies.

\begin{figure}[!ht]
   \begin{center}
      \includegraphics*[clip=true, width=6in]{pyFRET/heatmaps_errors.pdf}
      \caption{Heatmaps showing the effect on calculated FRET efficiency and peak area of varying the burst search thresholds L and M. A) Calculated FRET efficiency from DCBS analysis of FRET data. B) Calculated FRET efficiency from DCBS analysis of ALEX data. Missing values in B) are the result of insufficient events being retained for a Gaussian fit to be performed. C) Calculated peak area from DCBS analysis of FRET data. D) Calculated FRET efficiency from DCBS analysis of ALEX data. E) and F) show the Bayesian Information Criterion (BIC) calculated for each two-component fit of the FRET (E) and ALEX (F) data. A smaller (more negative) value of the BIC implies a better fit.}
      \label{fig:fig8_heatmaps}
   \end{center}
\end{figure}

It should be noted that, when used on FRET data, the APBS and DCBS burst search algorithms perform in an analogous manner to simple AND and SUM thresholding. Consequently, they retain the well-known disadvantages of these thresholding methods~\cite{murphy14}: specifically, APBS retains a zero-peak caused by donor-only labelled molecules, whereas DCBS is biased against extreme FRET efficiencies, distorting the FRET efficiency histogram. ALEX data does not display these biases, as the direct excitation of both fluorophores allows events to be selected independently of their FRET efficiencies. The effect of these biases in analysis of simple FRET data can be seen in Fig.~\ref{fig:fig6_Eplots} C, which overlays the calculated FRET efficiency curves for DCBS analysis of FRET and ALEX data. The curves for both AND-thresholded FRET data and DCBS burst selection of FRET data are distorted towards intermediate FRET efficiencies; ALEX data does not show this distortion. Consequently, ALEX is found to be a superior technique and should be used when available.

\subsection{Evaluating Gaussian Fitting}
\label{subsec:eval-gauu-fit-expl}
As part of pyFRET, a Gaussian mixture model was used to fit the FRET histograms with one or more Gaussian distributions. If it is assumed that a smFRET dataset contains values from $K$ fluorescent species with distinct FRET efficiencies, then the probability distribution of FRET efficiences, $E$ seen in the selected fluorescent events can be described using a linear combination of Gaussian distribtuions, 

\begin{equation}
\Pr(\text{params}|E) = \sum_{i=1}^K \phi_i~\mathcal{N}(\mu_i, \Sigma_i),
\label{eq:gmm}
\end{equation}

where $\Pr(\theta|E)$ is the probability of the set of parameter values $\theta$ given the observed vector of FRET efficiencies $E$, $\phi_i$ is the mixture weight of the $i$th component of the mixture and $\mathcal{N}(\mu_i, \Sigma_i)$ is the Gaussian distribution, with mean $\mu_i$ and covariance matrix $\Sigma_i$, that models the FRET efficiency of that fluorescent component. Using this model, the values of the parameters $\phi_i$, $\mu_i$ and $\Sigma_i$ can be optimised using an expectation maximisation algorithm~\cite{Guoshen2012}. The fits shown in the preceding figures use this mixture model. However, it is also important to understand the limitations of the fitting method and to demarcate the conditions under which multiple fluorescent populations could not be effectively distinguished using pyFRET. 

A combination of simulated and experimental data were used to benchmark the performance of the Gaussian fitting method when presented with multiple FRET populations. Firstly, FRET bursts drawn from a mixture of two FRET populations were simulated, with mean FRET efficiencies ranging from 0.1 to 0.9. These bursts were fitted with a two component Gaussian mixture model and the fits and FRET histograms overlaid. Example fits, shown in Fig.~\ref{fig:fig_benchmarking} A and B, demonstrate that where peaks are well separated (Fig.~\ref{fig:fig_benchmarking} A), they are correctly distinguished by the fitting protocol. Note that in A), where the two peaks are fitted well by the fitting procedure, the maximum height of each peak falls below the maximum height of the histogram that it overlays. This occurs because the histogram bins contain contributions from both distributions in the mixture, whereas the fits show the individual contributions from each distribution. However, when there is significant overlap between the FRET peaks (Fig.~\ref{fig:fig_benchmarking} B), there is insufficient information to distinguish the two populations, so the two Gaussians converge to very similar means, whose values lie in between the true mean values. This is an inherent problem with fitting multiple Gaussians to peaks that are poorly resolved, and is thus not a problem specific to the fitting procedure. Further details of the simulation procedure can be found in Section~\ref{subsec:benchmarking-gaussian}; the exact scripts used in the simulations can be found in Appendix~\ref{app:code_samples_GMM}.

\begin{figure}[!ht]
   \begin{center}
      \includegraphics*[clip=true, width=6in]{pyFRET/benchmarking.pdf}
      \caption{Benchmarking the Gaussian fitting process using simulated data. A) Fitting a two-component Gaussian mixture to a simulated dataset simulating a 1:1 ratio of FRET populations, with FRET efficiencies of 0.4 and 0.6. The two peaks are easily distinguished and fitted correctly. B) Fitting a two-component Gaussian mixture to a simulated dataset simulating a 1:1 ratio of FRET populations, with FRET efficiencies of 0.4 and 0.5. There is significant overlap between the two peaks, so they cannot be distinguished by the fitting algorithm. C) Heatmap showing the relationship between partner FRET efficiency and the error in calculated FRET efficiency. Gaussian fits of datasets where the FRET efficiencies of the two peaks are extremely close to each other are distorted towards an intermediate value. D) One-dimensional representation of the heatmap shown in C), showing the distortion in calculated FRET efficiency when the two components of the mixture have very similar FRET efficiencies.}
      \label{fig:fig_benchmarking}
   \end{center}
\end{figure}

To further quantify the limits of the fitting performance on two population mixtures, the discrepancy between the true and calculated FRET efficiencies was quantified for all peaks fitted from the simulated datasets, then plotted this discrepancy as a function of the FRET efficiency of the partner peak. The results are shown in Fig.~\ref{fig:fig_benchmarking} C and D. When the two peaks have extremely similar mean FRET efficiencies, namely separated from each other by less than their standard deviation, the fitting algorithm cannot effectively distinguish them, so both calculated values are distorted towards an intermediate value. However, when the two peaks are well separated, both peaks are fitted correctly (a typical p-value for the difference between the calculated and true distribution means, determined from a Student's T-test with 5000 samples is 0.97, showing no statistically significant difference), independent of the mean of the partner peak.

Finally, the performance of the Gaussian fitting on experimental datasets, consisting of a two-component mixture of DNA duplexes, was evaluated. Three duplexes were used for this experiment, namely the duplexes with a 4, 6 and 12 bp dye-dye separation, corresponding to a high-, intermediate- and low-FRET species respectively. The duplexes were mixed in either a 1:3 or 1:1 concentration ratio for data collection using ALEX. The data collection procedure is described in Sections~2.5.4 and~2.5.5. These data were analysed using the DCBS algorithm. The raw data were then re-binned into 1 ms time-bins and re-analysed using simple ALEX event selection. Additionally, the two photon channels ($D_D$ and $A_D$) that record donor and acceptor photons that arrive during donor excitation and hence correspond to the simple FRET part of an ALEX experiment were separated. These photon counts, from both the re-binned and the raw data, were used to compare the performance of the DCBS algorithm for FRET with simple FRET event selection using AND thresholding. Identified bursts were fitted with a two-component mixture of Gaussians. Note that the expected outcome of these experiments is a three-by-three grid of data points, corresponding to high, medium and low FRET efficiency duplexes, all measured at high, medium and low concentrations. Only DCBS on ALEX data (Fig.~\ref{fig:ratios} A) and B)) comes close to approximating this.

The FRET efficiencies and fractional peak areas determined using the burst search and simple thresholding analyses are shown in Fig.~\ref{fig:ratios} for the ALEX data and in Fig.~\ref{fig:ratios_FRET} for the simple FRET data. A comparison of these results shows several interesting points. Only DCBS on ALEX data can reliably identify three separate FRET efficiencies (Fig.~\ref{fig:ratios} A) and concentrations (Fig.~\ref{fig:ratios} B), independent of the concentration or FRET efficiency of the partner duplex. In contrast, the FRET efficiencies (Fig.~\ref{fig:ratios} C) and concentrations (Fig.~\ref{fig:ratios} D) identified by simple ALEX analysis of long time bins are distorted by the partner peak values.

The results are even more distorted when the simple FRET data is considered (Fig.~\ref{fig:ratios_FRET}). Compared with the ALEX data, the calculated FRET efficiencies are clearly distorted towards intermediate FRET efficiency values for both DCBS burst search (Fig.~\ref{fig:ratios_FRET} A) and simple thresholding (Fig.~\ref{fig:ratios_FRET} C) analyses. Furthermore, the population sizes calculated from the simple FRET data (Fig.~\ref{fig:ratios_FRET} B and D) are entirely scrambled, showing an extremely poor reflection of the underlying concentrations used. Concentrations in this experiment is a proxy for the relative population of states in typical smFRET experiment. Overall, from these four analyses of the same dataset, it is clear that the additional information available from an ALEX excitation regime is crucial to unbiased event selection and hence to accurate calculation of FRET efficiencies and population sizes. Furthermore, this additional information is best exploited using a burst search analysis rather than long time bins. Consequently, it is found that the performance of the Gaussian fits are only as good as the prior event selection, and that, when there is explicit access to acceptor excitation information, burst search across short bins outperforms simple thresholding on long bins.

\begin{figure}[!ht]
   \begin{center}
      \includegraphics*[clip=true, width=6in]{pyFRET/ratios.pdf}
      \caption{Using pyFRET to fit multiple fluorescent populations from ALEX data. ALEX data from mixtures of two DNA duplexes were analysed using pyFRET DCBS (A) and B)) or simple ALEX thresholding on re-binned data (C) and D)), then fitted using a two-component Gaussian mixture model. Three duplexes were used, to provide high-, intermediate- and low-FRET peaks. A) and C) Plot of FRET efficiency vs fractional population, coloured according to the dye-dye separation in the duplex. B) and D) Plot of FRET efficiency vs fractional population, coloured according to the population size. These figures are discussed in detail in Section~\ref{subsec:eval-gauu-fit-expl}.}
      \label{fig:ratios}
   \end{center}
\end{figure}

\begin{figure}[!ht]
   \begin{center}
      \includegraphics*[clip=true, width=6in]{pyFRET/ratios_FRET.pdf}
      \caption{Using pyFRET to fit multiple fluorescent populations from FRET data. FRET data from mixtures of two DNA duplexes were analysed using pyFRET DCBS (A) and B)) or simple AND thresholding on re-binned data (C) and D)), then fitted using a two-component Gaussian mixture model. Three duplexes were used, to provide high-, intermediate- and low-FRET peaks. A) and C) Plot of FRET efficiency vs fractional population, coloured according to the dye-dye separation in the duplex. B) and D) Plot of FRET efficiency vs fractional population, coloured according to the population size. These figures are discussed in detail in Section~\ref{subsec:eval-gauu-fit-expl}.}
      \label{fig:ratios_FRET}
   \end{center}
\end{figure}

\clearpage

\subsection{Benchmarking the RASP Algorithm}    
The pyFRET implementation of the RASP algorithm was benchmarked using data collected from a 1:1 mixture of 6 bp (high FRET) and 12 bp (low FRET) duplexes over a period of 10 hours (600 minutes). The parameters used in RASP analysis are shown in Table~\ref{tab:RASP}. RASP allows selection and analysis of bursts that occur within a time interval $\Delta$T after bursts with FRET efficiency in the range $E_{min} - E_{max}$. The correct performance of the RASP algorithm is demonstrated by analysing bursts that occur in 1 ms intervals, centred at the stated times, following a high FRET burst. The resultant FRET histograms are shown in Fig.~\ref{fig:fig9_RASP}. At short recurrence intervals ($T_{max} < 8$ ms), not only do the great majority of events show a high FRET efficiency, but the number of events is greatly increased compared with longer recurrence intervals, demonstrating the enrichment of these events by molecules that have diffused back into the confocal detection volume. 


\begin{figure}[!ht]
   \begin{center}
      \includegraphics*[clip=true, width=6in]{pyFRET/RASP_performance.pdf}
      \caption{RASP analysis of FRET data from a mixture of a high-FRET and low-FRET duplex. The black line shows the baseline peak areas for the two duplexes. Other lines show the FRET histograms generated from bursts that occurred at the indicated time following a high-FRET burst. The Recurrence Interval $\Delta$T was 1 ms, centred at the times shown in the legend. Note the greatly increased peak area for high-FRET bursts, demonstrating the recurrence of high-FRET events as molecules diffuse back into the confocal volume.}
      \label{fig:fig9_RASP}
   \end{center}
\end{figure}


\section{Availability and Future Directions}
\label{sect:availability}
pyFRET is available to download under an open source BSD licence from the Python Package Index (\url{https://pypi.python.org/pypi/pyfret0.1.0}). Documentation can also be found here, whilst a more extensive tutorial, including example scripts, can be found online (\url{http://pyfret.readthedocs.org/en/latest/tutorial.html}). Example scripts that reproduce the evaluation described in this chapter can be found in Appendix~\ref{app:code_samples}

smFRET is a fast-developing and active research field and although pyFRET provides the core tools for analysis of smFRET data, not all the algorithms used in analysis of smFRET data are currently implemented. In particular, pyFRET is currently not able to parse files generated using the picoQuant instrumentation and neither stochastic denoising nor photon distribution analysis~\cite{kalinin2007, antonik2006, santoso10, torella11} are implemented. I am keen to extend the functionality of pyFRET and happy to work with others to enable their use of and contribution to the pyFRET library.

\clearpage

\section{Conclusions}
This chapter presented pyFRET, a versatile open source library for analysis of smFRET data. In addition to demonstrating the usage of these event selection and fitting algorithms, the performances of these algorithms were thoroughly evaluated. During this evaluation, it was found that burst search algorithms, when combined with ALEX data collection, are robust to changes in the values of parameters T, L and M; and can accurately determine the population sizes and FRET efficiencies of a variety of experimental datasets. On the other hand, neither simple thresholding nor the sophisticated burst search algorithms could overcome the biases introduced through event selection in simple FRET datasets. As a consequence, there has been considerable interest in the development of unbiased event selection and analysis tools for the analysis of smFRET datasets. The following chapter (Chapter~\ref{chap:inference}) describes the development and evaluation of a Bayesian analysis tool that can perform a complete analysis of time-binned, smFRET data in a single step, with no intermediate event selection or denoising.  

%It should be noted that  ALEX data collection, on the other hand, does not suffer from these biases, so is able to analyse a wide variety of smFRET datasets with considerable accuracy. Furthermore, the burst search algorithms, especially when combined with ALEX data collection, are robust to changes in the values of parameters T, L and M, ensuring that conclusions drawn from smFRET data are not influenced by the thresholding parameters used.

%One of the principal

%\section*{Tables}








