\chapter{Bayesian Inference of Oligomer Sizes Using Single Molecule FRET}
\label{chap:sizing}

\section{Overview}
This chapter describes the extension of the Bayesian model based inference described in the previous chapter to to problem of oligomer sizing in the study of protein aggregation. The introductory section opens with a brief overview of the diseases of protein aggregation as motivation for the study of small aggregates. We then describe prior research into protein aggregation using single molecule microscopy, paying particular attention to the assumed relationship between aggregate size and emitted photons. Next, we describe sources of heterogeneity that complicate this relationship and describe acousto-optic modulation as a method to reduce heterogeneity. Finally, we introduce the Holliday Junction~\cite{holliday1964} as a model oligomer, which can be used to test sizing tools.

This chapter presents results in two main areas. Firstly, we describe the adaptation of our FRET emission model to a simplified model of oligomer emission. We present a number of simple simulations modeling smFRET experiments on oligomers of known size, or in mixures of known size distribution. We show that when the emission rate is assumed to be poisson, the number of emitted photons is a good metric for oligomer size. However, when the emission rate is over-dispersed, the number of photons emitted is no longer dominated by the number of fluorophores present, so is a bad measure of oligomer size. We then use fluorescently labelled DNA Holliday Junctions as model oligomers of known size. Using these oligomer models, we next show, through comparison of simulations with experimental data from single colour single molecule fluorescence experiments, that the photon emission distribution is indeed over-dispersed. 

Secondly, we present an attempt to fit the parameters of our model using both simulated and experimental data. We show that accurate inference of even the mean monomer brightness is difficult, because the relationship between fluorophore emission and confocal occupancy is challenging to decouple. As a consequence, we show that our model of the emission process unfortunately cannot be used to reliably infer the oligomer size distribution, with the result that we cannot accurately determine oligomer size from the number of observed photons.

Finally, we attempt to understand the sources of the emission broadening. We undertake a photobleaching analysis of our synthetic oligomers. From these data, we show that the dye Alexa Fluor 488 undergoes reversible photoblinking on a time-scale comparable with the dwell time in the confocal volume, identifying an important additional source of emission heterogeneity. We also discuss the effect of unequal excitation power on photon emission and suggest experimental modifications that could mitigate this problem. 

%
%Following this introduction, we describe the adaptation of our FRET emission model to a simplified model of oligomer emission. We describe how this model can be used to infer the brightness of a monomer against a noisy, heterogeneous background. The results presented in this chapter are as follows. Firstly, we present an analysis of some simulated datasets, demonstrating that the complex relationship between size and photon emission leads to extremely broad photon distributions that make accurate sizing difficult. Through a comparison of experimental and simulated datasets, we show that the gamma-poisson mixture model is a good model for the emission process. Secondly, we present an attempt to fit the model parameters using both simulated and experimental data. We show that accurate inference of even the mean monomer brightness is difficult, because the relationship between fluorophore emission and confocal occupancy is challenging to decouple. 

%To better understand the relationship between size and emision, we collect single molecule fluorescence data from DNA Holliday junctions labelled with between one and four fluorophores. We analyse this data using both our inference tool and standard thresholding techniques, showing that neither is sufficient to accurately describe the data.

%Finally, we use a combination of TIRF microscopy and accousto-optic modulation to probe, and attempt to reduce, the sources of emission heterogeneity. We demonstrate that both the non-uniform excitation profile and the presence of significant photoblinking contribute to our inability to correctly size oligomers based on their fluorescence emission. 

\section{Introduction}
\subsection{Diseases of Protein Aggregation}
Several human diseases, such as Alzheimer's Disease~\cite{ueda1993} and Parkinson's Disease~\cite{spillantini1997} are diseases of protein aggregation. The neurodegenerative Alzheimer's Disease is characterised by loss of neuronal cells and the presence of large protein aggregates both in the extra-cellular space and within cells. The extra-cellular deposits are primarily composed of polymers of the peptide Amyloid Beta (hereafter A${\beta}$)~\cite{Noe2004}, whilst the intracellular aggregates, named neurofibrillary tangles, are dominated by the protein Tau~\cite{goedert1989}. Similarly, at the cellular level, Parkinson's Disease is charaterised by the presence of Lewy Bodies~\cite{Gibb1988}, solid depositions of protein, predominantly the intrinsically disordered protein $\alpha$-synculein~\cite{spillantini1997}, inside nerve cells.

% Need figures: micrographs of lewy body, nf tangle, etc 

In both Alzheimer's Disease and Parkinson's Disease, the protein aggregates, when examined at the microscopic level, are found to have a similar structure: long, polymeric chains of proteins form fibrils with a regular structure comprised of inter-molecular (cross) beta sheets. These amyloid fibres are also found in patients suffering from other forms of neurodegenerative disease, such as Transmissible Spongiform Encephalopathy~\cite{Wells1987} and Fatal Familial Insomnia~\cite{Goldfarb1992}, as well as in other human diseases that do not include neurodegerative symptoms~\cite{Clark1988} and some animal diseases~\cite{Safar1993}. 

% Need figures: structure of amyloid (diagram); structure of amyloid (TEM).

The role that amyloidosis plays in neurodegenerative disease is currently not well understood. Heritable mutations, such as the A30P and A53T mutations found in $\alpha$-synculein, which predispose the peptide to aggregate formation in vitro~\cite{Conway1999} are associated with early onset disease. However, it is curently unknown whether there is a common mechanism of cytotoxicity associated with the structurally similar oligomeric species found in various diseases of protien aggregation,~\cite{Haas2007}; whether the accumulation of amyloid aggregate overwhelms the cellular degradation and molecular chaperone pathways~\cite{Muchowski2005}, or whether the mechanisms of toxicity are complex and disease specific~\cite{Benilova2012}.      

\subsection{Studying Protein Aggregation}
\label{subect:aggregation}
As a consequence, understanding the chemical pathway of protein aggregation, particularly of disease-associated peptides such as A${\beta}$ and $\alpha$-synculein, is of considerable biological interest. Current understanding is that aggregation follows a ``nucleated growth" mechanism~\cite{chiti2006}. Under the nucleated growth mechanism, there is an initial lag phase, during which a few protein molecules convert from their native structure to an aggregation nucleus. Following their formation, fibril growth occurs rapidly via polmerisation from these molecular seeds (see Fig.~\ref{fig:aggregation_pathways}). Analytical solution of the kinetic equations of aggregation suggests that secondary nucleation via templating dominates the rate of aggregation~\cite{knowles2009}. This theoretical result has been used to explain the differing experimentally observed aggregation rates of the two amyloidogenic peptides A$\beta$-40 and A$\beta$-42~\cite{cohen2013, meisl2014}.

\begin{figure}
   \begin{center}
      \includegraphics*[clip=true, width=6in]{sizing/AmyloidMechanisms.jpg}
      \caption{Schematic showing pathways of amyloid aggregation. Figure from~\cite{Giurleo2008} and used with permission.}
      \label{fig:aggregation_pathways}
   \end{center}
\end{figure}


To fully understand the protein aggregation pathway and the nature of the biochemical species involved, it is necessary to study both the kinetics of the aggregation reaction and the biophysical properties of the different species on the aggregation pathway. Owing to its relative abundance, the native structure of the monomeric species can be characterised relatively effectively using, for example, NMR solution studies~\cite{Dedmon2005}. Similarly, excellent insight into the morphology of mature fibrils can be obtained through experimental techniques such as TEM microscopy~\cite{Gras2011} and SAXS~\cite{Langkilde2009}. Furthermore, the chemical Thioflavin T (ThT hereafter)(Fig.~\ref{fig:ThT} A), which is weakly fluorescent when free in solution, becomes several orders of magnitude more fluorescent on binding to the beta sheet structure of an amyloid fibril, allowing it to be used to study the kinetics of fibril formation (\cite{Cohen2012}, Fig.~\ref{fig:ThT} B).

\begin{figure}
   \begin{center}
      \includegraphics*[clip=true, width=4in]{sizing/tht_fig.pdf}
      \caption{ThT fluorescence is used to follow the kinetics of amyloid formation. A) The molecular structure of ThT. When free in solution, rotation around the central bond between the benzylamine and benzathiol rings within the molecule enables twisted intramolecular charge transfer in the excited state, and a low quantum yield. On binding to extended beta-sheet structures, the rotation is prevented and the fluorescence quantum yield increases significantly. B) Using ThT fluorescence to track the kinetics of amyloid formation reveals a characteristic sigmoidal curve, displaying an initial lag phase, followed by a rapid growth phase that plateaus once the aggregation reaction reaches equilibrium. B) is adapted from \cite{Cohen2012}.}
      \label{fig:ThT}
   \end{center}
\end{figure}

\paragraph{Using smFRET to Identify Oligomers}
Unfortunately, the all important oligomeric species involved in nucleation remain relatively elusive to scientific study. Although the presence of oligomers can be detected through ELISA assay~\cite{Schmidt2005} and synthetic oligomers can be prepared in vitro~\cite{Murray2003}, direct charaterisation of oligomeric species is hindered by the difficulty of accurately detecting oligomers against a background of monomeric and fibrillar structures. One promising method of oligomer observation is through the use of single molecule fluorescence studies.

Previous work using two colour single-molecule microscopy has been able to both track the kinetics of aggregation of $\alpha$-synuclein and to identify different oligomeric states~\cite{cremades2012}. This research used $\alpha$-synuclein monomers labelled with one of the fluorescent dyes Alexa Fluor 488 and Alexa Fluor 647. Mixtures of monomers labelled with the dyes were incubated under fibrillizing conditions and then subjected to single molecule analysis at time points distributed throughout the duration of the aggregation reaction. AND-based thresholding was used to identify oligomeric species against an overwhelming background of monomers, as only oligomers carrying at least one of each dye type would be able to display photon emission in both donor and acceptor channels. These experiments identified two fluorescent species, displaying different FRET efficiencies. Moreover, the high-FRET species emerged later in the aggregation reaction, was associated with oligomers of larger size and was further demonstrated to show increased resistance to Proteinase-K degradation, suggesting an interconversion between two oligomeric species as the aggregation progressed.

\subsection{The Relationship Between Size and Photon Emission is Complex}

% I need to find a way to describe the sizing as a limitation, without seeming to belittle the whole paper ...

This research provides a fascinating insight into the biophysical changes that occur during amyloid aggregation. However, one limitation is that the method used for determining the size of an observed oligomer is essentially a simple heuristic. Oligomer sizes were determined by comparison with the brightness of a monomer event, correcting for the increased dwell-time of larger molecules in the confocal volume, but otherwise assuming a linear relationship between oligomer size and photon emission rate~\cite{orte08}:

\begin{equation}
\text{size} = \frac{2\cdot I_D + \gamma^{-1} I_A}{I_{\text{monomer}}}
\label{eq:size_linear}
\end{equation}

for $I_D$ and $I_A$ the intensities (number of photons) in the donor and acceptor channels respectively, $\gamma$ the instrumental gamma factor and $I_{\text{monomer}}$ monomer event brightness, determined from fluorescent bursts that consisted of photons of a single colour only by selecting events for which $I_D \geq T_D$ but $I_A < T_A$.

However, as described in the previous chapter (Section~\ref{subsec:justification}), not all fluorescently labelled molecules passing through the confocal volume result in emission of the same number of photons. Multiple factors, including the confocal dwell time, the pathway taken through the gaussian laser beam and the effect of photobleaching during excitation result in broadening of the photon emission distribution becoming super-poissonian. The emision behaviour is well-approximated by a gamma-Poisson mixture model~\cite{chen1999}, as we use in our inference-based analysis tool~\cite{murphy14}. Single molecule fluorescence studies of oligomeric species are subject to similar behaviours, meaning that the relationship between oligomer size and burst brightness is complex and non-linear. 

%The rest of this chapter describes our attempts to improve our understanding of this relationship, using a combination of simple simulations and experimental observations of model oligomers of known size. This chapter presents results in two main areas. Firstly, we describe a number of simple simulations modeling smFRET experiments on oligomers of known size, or in mixures of known size distribution. We show that when the emission rate is assumed to be poisson, the number of emitted photons is a good metric for oligomer size. However, when the emission rate is over-dispersed, the number of photons emitted is no longer dominated by the number of fluorophores present, so is a bad measure of oligomer size. We next show, through comparison of simulations with experimental data from single colour single molecule fluorescence experiments on model oligomers of known size, that the photon emission distribution is indeed over-dispersed. Further, we show that our model of the emission process unfortunately cannot be used to reliably infer the oligomer size distribution, with the result that we cannot accurately determine oligomer size from the number of observed photons.

%Finally, we attempt to understand the sources of the emission broadening. We first attempt to reduce emission heterogeneity by reducing the excitation heterogenity within the confocal detection volume. Unfortunately, this is unsuccessful in sufficiently reducing the emission heterogeneity. Subsequently, we undertake a photobleaching analysis of our synthetic oligomers. From the photobleaching data, we show that the dye Alexa Fluor 488 undergoes reversible photoblinking on a time-scale comparable with the dwell time in the confocal volume, identifying an important additional source of emission heterogeneity. 
 

%\subsection{The Effect of Confocal Excitation Heterogeneity on Photon Emission}
%XXX This needs refs! And state equations! XXX
%One of the factors controlling the brightness of fluorescent emission events is the power of the laser used to excite the fluorophore. Each time a fluorescently labelled molecule passes through the exciting laser beam, it is bombarded with photons of the correct wavelength to allow excitation of the fluorophore from the ground state to the excited state (Fig?XXX). The total number of photons emitted by the fluorophore depends on the number of successful excitation-emission cycles undergone. In bulk solution, the relationship between laser power and fluorescence emission shows a charateristic inverted U-shape~\ref{fig:power_fluorescence}. This is caused by two opposing effects. At low excitation powers, the fluorophore spends most of its time in the ground state, so the photon emission rate is limited by the number successful excitation-emission cycles it undergoes whilst in the laser beam. Increasing the laser power increases the rate at which photons interact with the ground state fluorophore, so increases the number of potentially successful excitation-emission cycles, resulting in an incresed rate of photon emission. However, as the laser power is further increased, fluorophores in the excitation volume spend more time in the excited state. Now the limiting factor is photobleaching, caused by the excited state of the fluorophore accepting a second photon and undergoing a permanent conformational change. Consequently, further increasing the laser power results in a reduction in the number of emitted photons, as the increased probability of bleaching reduces the average number of successful absorption-emission cycles each fluorophore undergoes before permanent bleaching occurs. 

%A single-molecule fluorescence experiment is typically performed in the first regime, where photon absorption limits the number of successful emission cycles. However, the laser itself has an approximately gaussian beam profile. This means that, the pathway that individual molecules take through the confocal volume affects the number of photons emitted: a fluorophore that passes through the centre of the confocal volume will be exposed to a higher number of exciting photons than a fluorophore that passes only through the edge of the laser beam~\ref{fig:beam_profile}. Consequently, although the photon emission distribution with a uniform excitation field is a poisson proces~\cite{???}, the photon emission distribution under gaussian laser excitation is over-dispersed and hence can be modeled as a gamma-poisson mixture distribution~\cite{chen1999}. This has considerable implications for our ability to accurately size oligomeric species based on the number of photons emitted in an excitation event.

%\subsection{Controlling Confocal Excitation Heterogeneity}
%To understand the role that confocal excitation heterogeneity plays in broadening the photon emission distribution, we designed a novel microscope set-up that allowed controllable broadening of the confocal excitation volume in one dimension, from a diffraction limited spot to a line of uniform excitation that over-filled the detection volume. The optical set-up, which is shown in Fig.~\ref{fig:uniform_excitation}, uses rapid acousto-optic deflection of the laser beam to effectively broaden the area of excitation, increasing the uniformity of the exciation power within one dimension of the detection volume.

%The acousto-optic effect is the alteration of the permittivity of a material caused by mechanical strain~\cite{???}. In acousto-optic deflection, a piezo-electric transducer converts applied electrical signal into acoustic waves.  These accoustic waves, when applied to a crystal, create longitudinal(?) vibrations, generating strain that alters the crystal's permittivity to certain frequencies of light. Experimentally, this results in an alteration in the crystal's deflection of incident light~\cite{???}. Varying the frequency of the applied electric field varies the speed with which the permittivity of the crystal changes; varying its amplitude alters the degree of deflection. (???)  

%XXXohfuckhelpidontdophysicsXXX    


\subsection{The DNA Holliday Junction as a Model Oligomer}
To understand the relationship between the number of fluorescent labels and the number of emitted photons and hence to evaluate the accuracy of oligomer sizing using single molecule fluorescence, it was necessary to have access to synthetic oligomers carrying a known number of fluorescent labels. For this model oligomer, we used the DNA Holliday Junction~\cite{holliday1964}. The Holliday Junction is a four-way structure that is observed biologically during DNA recombination~\cite{potter1976}. Synthetic Holliday junctions can be prepared using four partially complementary strands of DNA (Fig.~\ref{fig:holliday_junction}) and have frequently been used to study junction dynamics using single molecule fluorescence~\cite{mckinney03, uphoff2010, Hyeon2012}. These synthetic structures provide an ideal model for small oligomers in fluorescence studies, as each strand can be independently labelled with a fluorescent dye prior to construction of the junction, enabling precise control of the number of fluorescent dyes present (Fig.~\ref{fig:holliday_junction_labelling}). We prepared synthetic Holliday junctions labelled with between one and four fluorescent dyes as described below. The sequences of the four arms used are shown in Table~\ref{tab:table1_DNAsequences} These synthetic species were then used in single molecule fluorescence studies to probe the relationship between oligomer size and photon emisson.

\begin{figure}
   \begin{center}
      \includegraphics*[clip=true, width=3in]{sizing/holliday_junction.pdf}
      \caption{Schematic of a synthetic Holliday junction. The four arms, named B, H, R and X are shown in red, blue, yellow and green respectively.}
      \label{fig:holliday_junction}
   \end{center}
\end{figure}

\begin{figure}
   \begin{center}
      \includegraphics*[clip=true, width=4in]{sizing/holliday_junction_labelling.pdf}
      \caption{Schematic showing the four labelling states of our synthetic Holliday Junction construct. Shown clockwise from top left: The monomer model has only Arm B labelled (top left); the dimer has both arms B and X labelled (top right); the trimer model has labels on arms B, H and R (bottom right); in the tetramer model all four arms are labelled (bottom left).}
      \label{fig:holliday_junction_labelling}
   \end{center}
\end{figure}

\section{Theory}

This section describes the extension of our generative model of the smFRET experiment to the case of fluorescently labelled oligomers, carrying multiple fluorescent dyes. Here, we describe two different models of photon emission from a labelled oligomer. First, we describe a simple poisson model in which sources of distribution broadening are not considered. We then extend this model to a gamma-poisson mixture model, which includes overdispersion. We show, through comparison of simulations and experimental data that the gamma-poisson mixture is a more appropriate model for oligomer fluorescence.

For simplicity, we consider only molecules labelled with a single dye type (experimentally, either Alexa Fluor 488 or Alexa Fluor 647), allowing FRET effects to be ignored. Several other assumptions are also made, namely that the relative concentration of oligomeric species is modelled by a geometric distribution and that all monomers carry a fluorescent dye (a labellling efficiency of 100\%.) Further, we consider only events binned on the order of the dwell time in the confocal volume and do not explicitly consider either the effect of photobleaching partway through a bin or changes in the dwell time caused by slower diffusion of larger molecules. Details of the model, and the roles of these assumptions are given below.

\subsection{A Simple Poisson Model of Oligomer Photon Emission}
In the simplest fluorescence experiment from an oligomeric species, the fluorescently-labelled oligomers diffuse freely through the confocla detection volume.  When a molecule diffuses into the confocal volume, the laser excites the attached fluorophore(s) and photons are emitted (Fig.~\ref{fig:oligomers}. Typically, both donor and acceptor fluorophores would be used to allowing identification of oligomeric species based on the presence of coincident donor and acceptor bursts. However, we consider a simplified experiment in which monomers are labelled with one dye type only. This enables us to ignore the effect of FRET on the number of photons observed and to consider only the total number of photons emitted in a burst. We also assume full labelling of monomers (100 \% labelling efficiency), so that the number of fluorophores present is equivalent to the oligomer size (in monomers). Under these conditions, we expect intuitively that the number of emitted photons increases with the number of fluorophores present. As for dual-labelled monomers, we consider photons binned into time-bins of length on the order of the dwell time in the confocal volume; we expect the great majority of time-bins to contain noise photons only.

\begin{figure}
   \begin{center}
      \includegraphics*[clip=true, width=6in]{sizing/multimer_emission.pdf}
      \caption{Schematic showing the problem of determining oligomer size from the number of observed photons. A) If we observe an average of 10 photons when a monomer diffuses through the confocal volume, how many more should we expect if a tetramer takes the same pathway through the confocal volume? B) Possible relationships between the number of fluorophores and the average number of photons observed. If there is a linear relationship (blue line) then we expect a tetramer to emit an average of 40 photons; if the relationship is not linear, we may expect fewer photons (red and green lines).}
      \label{fig:oligomers}
   \end{center}
\end{figure}

We model this simple experiment as a sequence of measurements ($f_D$) of the number of photons observed in the fluorescence emission channel. Each time-bin is treated as an independent and identically distributed sample from a set of random variables describing the dataset. Each data point ($f_D$) in the data stream is the sum of noise photons and, where an oligomer is present, some photons from a fluorescent event. 

The number of noise photons is drawn from a Poisson distribution with rate parameter $\lambda_D$. The probability of observing $n_D$ noise photons in the donor channel is then:
\begin{equation}
n_{D} \sim \text{Poisson}(n_{D}; \lambda_{D}) = \frac{\lambda_{D}^{n_{D}}}{n_{D}!}e^{-\lambda_{D}}
\end{equation}

In addition to noise, each observation may contain photons from one or more fluorescently labelled oligomers.  As before, the number of molecules present in the excitation volume, $n_{\text{prot}}$ can be described using a Poisson distribution with rate parameter $\lambda_{\text{prot}}$:
\begin{equation}
n_{\text{prot}} \sim \text{Poisson}(n_{\text{prot}}; \lambda_{\text{prot}}) = \frac{\lambda_{\text{prot}}^{n_{\text{prot}}}}{n_{\text{prot}}!}e^{-\lambda_{\text{prot}}}
\label{eq:nprot}
\end{equation}

However, now we must also consider the oligomer size. In the absence of evidence to the contrary, we assume that fluorophores on an oligomer are non-interacting, such that each fluorophore is excited independently and does not experience quenching effects from the proximity of other fluorophores. We begin with a simple, purely poisson model of oligomer fluorescence. We model each fluorophores as displaying poisson emission with rate parameter $\lambda$. As the sum of poisson random variables is itself a poisson random variable~\cite{Lehmann1986}, an oligomer of size $n$ can be described as having an emission rate of $n \cdot \lambda$. Therefore, the dditional photons present in a time-bin contributued by an oligomer of size $n$ is given by:

\begin{equation}
c_D \sim \text{Poisson}(c_D; n \cdot \lambda)
\end{equation} 

The total number observed photons in that time bin, $f_D$, is then the sum $n_D + c_D$.

Given this simple model, we can easily generate multiple datasets with different properties. Below, we describe generation of simulated datasets from oligomers of known size, or from mixtures of oligomer sizes of known size distributuion. However, first we introduce the more complex gamma-poisson mixture model that is a direct extension of the model described in the previous chapter.

\subsection{A Gamma-Poisson Mixture Model of Oligomer Photon Emission}
The previous section described a simple poisson model of photon emission from labelled oligomers. However, as discussed in the previous chapter, we have found a gamma-poisson (negative binomial) distribution to be a better model for the overdispersion seen in emission from dual-labelled monomers~\cite{murphy14}. This section describes the extension of this model of fluorescence emission to the case of labelled protein oligomers.

The system modelled is equivalent to that described in the previous section. The noise distribution and probability of confocal volume occupancy are unchanged. However, now we choose to include over-dipsersion in our model of fluorophore excitation, by drawing event specific emission parameters from a gamma distribution, moving from a single-step model of emission to a two stage emission process. 

Firstly, given that a fluorescently labelled molecule is present in the confocal volume, we determine a rate of donor emission, $\lambda$ for monomers in that specific molecule as a random sample from a gamma distribution with shape parameter $k_D$ and mean $\lambda_B$ (Eq.~\ref{eq:gamma_size}). This captures the variation in the number of photons emitted by a molecule as a result of the diffusion path taken through the confocal volume and the effect of individual fluorophores photobleaching partway through an observation.

\begin{equation}
\lambda \sim \text{Gamma}(\lambda; k_D, \theta) =  \frac{1}{\Gamma(k_D) \theta^{k_D}} \lambda^{(k_D - 1)} e^{-\frac{\lambda}{\theta}} \quad\text{ for } \theta = \lambda_B / k_D
\label{eq:gamma_size}
\end{equation} 
Here, $\Gamma$ is the Gamma function. 

In the second step, we consider the oligomer size. As above, we consider labelleing efficiency to be 100 \%, such that an oligomer containing $n$ monomers is assumed to have $n$ attached fluorescent dyes. We assume these dyes to be non-interacting. Furthermore, although we are now considering the impact of the pathway through the confocal volume on the number of photons emitted by a fluorophore, the size of an oligomer is still small relative to the size of the confocal volume. Hence, we assume that all fluorophores on a specific oligomer have the same rate of donor emission, $\lambda$, determined according to equation~\ref{eq:gamma_size}. Consequently, the number of photons emitted by an oligomer of size $n$ with emission rate $\lambda$ is given, as before by

\begin{equation}
c_D \sim \text{Poisson}(c_D; n \cdot \lambda)
\end{equation}  

and the total number of observed photons, $f_D$ is then the sum $n_D + c_D$. Note, however that whereas in the simple poisson model $\lambda$ was a constant, now $\lambda$ itself is a random variable, which is drawn afresh from equation~\ref{eq:gamma_size} for each oligomer emission event.

The rest of this chapter describes the comparison of simulated datasets generated using these two models with real data from model oligomers of known size and from true protein aggregation experiments with unknown oligomer size distributions. In the next section, we summarize the experimental techniques used to collect single molecule fluorescence data from aggregation timecourses; the preparation of labelled Holliday Junctions as model oligomers; and the modification of the confocal excitation set-up to modify the confocal excitation field. 

\section{Experimental Methods}
%\subsection{Labelling of Protein Monomers}
%\subsection{Protein Aggregation Experiments}
\subsection{Preparation of DNA Holliday Junctions}
\paragraph{Holliday Junction Preparation}
\emph{Holliday Junction purification was carried out by Vladimiras Oleinikovas under the supervision of the author.}
Four oligonucleotides labeled at the 5-prime end by the fluorescent dye Alexa Fluor 488 and with the correct sequence to anneal to form a Holliday Junction were purchased (atdbio), along wih equivalent unlabelled strands (Sigma Life Science). Their sequences are given in Table~\ref{tab:table1_DNAsequences}. The four strands required to generate a Holliday junction were combined in Tris buffer (10 mM, pH 8.0, 50 mM NaCl), to give a final total DNA concentration of 5 $\mu$M in a total volume of 20 $\mu$L, using an equimolar concentration of the four DNA strands. The single-stranded DNAs were annealed into the Holliday Junction by heating to 95$\circ$C for 10 minutes, followed by slow cooling overnight to 25$\circ$C.

\begin{center}
\begin{table}[!ht]
\caption{DNA sequences of the four arms of the Holliday Junction. Each sequence is shown in the 3' - 5' direction/ {\bf 5} is a maleimide linkage to Alexa Fluor 488, present in the fluorescently labelled DNA strands, but absent in their unlabelled partners.}
%\begin{adjustwidth}{-2.25in}{0in} % Comment out/remove adjustwidth environment if table fits in text column.
\begin{tabular}{|l|l|}
\hline
{\bf Arm} & {\bf Sequence}\\ \hline
B & CCCTAGCAAGCCGCTGCTACGG{\bf 5} \\
H & CCGTAGCAGCGAGAGCGGTGGG{\bf 5} \\
R & CCCACCGCTCTTCTCAACTGGG{\bf 5} \\
X & CCCAGTTGAGAGCTTGCTAGGG{\bf 5} \\ \hline
\end{tabular}
\label{tab:table1_DNAsequences}
%\end{adjustwidth}
\end{table}
\end{center}


\paragraph{Holliday Junction Purification}
\emph{Holliday Junction purification was carried out by Vladimiras Oleinikovas under the supervision of the author.}
After annealing, the Holliday Junctionss were purified from the reaction mixture using gel electrophoresis on an 8\% TBE acrylamide gel in TBE buffer (Novex, Life Technologies Ltd.). The complete Holliday Junction was identified by visualising the bands using low intensity (PMT 300V) excitation at 526 nm and comparison with a 10 bp DNA ladder. The band corresponding to the complete Holliday Junction was cut from the gel and eluted into 200 $\mu$L of Tris buffer by passive difusion overnight. Holliday Junctions were then stored in the dark at 4$\circ$C until required.

\subsection{Simple FRET Measurements of DNA Holliday Junctions}
Labelled Holliday Junctions were diluted to a concentration of 50 pM in TEN buffer (10 mM Tris, 1mM EDTA, 100 mM Nacl), pH 8.0, containing 0.01 \% Tween-20. FRET data were collected for 15 minutes using continuous excitation at 488 nm at a power of 80 mW. Collected photons were binned online in intervals of 1 ms and stored in files of 10000 bins. 

%\subsection{Flattening the Confocal Volume Using Acousto-Optic Deflection: A Modified Single Molecule Fluorescence Microscope}
The microscope used for experiments into the effet of unequal excitation was similar to that described in previous chapters. However, a number of modifications were used to allow alteration of the detection volume. Firslty, a voltage controlled oscillator (DRFA10Y-B-0, PhotonLines) was coupled via an RF amplifier (AMPA-B-30, PhotonLines) to an acousto-optic modulator (MT80-A1-VIS, AA Optolectronincs). The modulator was placed in the beam path of the 488 nm laser () and a triangular wave, provided using a signal generator (), was used to deflect the beam in the transverse direction at a speed of 100 kHz. Modifying the amplitude of the wave allowed the degree of deflection to be precisely controlled. This enabled us to quantify the effect of unequal excitation power on the photon emission distribution. The AO modulator deflects only the first harmonic of the 488 nm laser, which is spatially separated from the zeroth and other order beams. Consequently, we direct only this beam into the back port of an inverted microscope (Nikon Eclipse Ti-U). This harmonic corresponds to approximately 10 \% of the total power output of the laser.

%\subsection{Preparation of Microfluidic Channels}
%\paragraph{Preparation of the Master Plate}
%\emph{Master Plate preparation was carried out by Mathew Horrocks. His master plates were used with permission.}
%Master plates for the microfluidic devices were prepared by spin-coating SU-8 2025 photoresist (MicroChem) onto silicon wafer (diameter: 76.2 mm, Compart Technology Ltd.) at 800 rpm for 5 s followed by ramping to 3000 rpm at an acceleration of 300 rpm$s^{-1}$ for 60 s to give a final film thickness of 25 $\mu$m. After spinning, the wafer was prebaked (1 min at 65 $^{\circ}$C, then 3 min at 95 $^{\circ}$C and finally 1 min at 65 $^{\circ}$C), and then exposed to UV light through the mask on a mask aligner (MJB4, SUSS Microtec). After postbaking and development,the master was hard-baked for 60 s at 170 $^{\circ}$C.

%\paragraph{Preparation of Individual Devices}
%A 10:1 (w/w) ratio of PDMS and curing agent (Sylgard 184, Dow Corning) was prepared and poured over the master, degassed and then baked overnight at 75 $^{\circ}$C. The devices were cut and peeled off the master. Individual devices were prepared by slicing the PDMS into sections containing approximately six microfluidic channels, punching inlet and outlet holes for each channel using a biopsy punch, and then plasma bonding the PDMS onto plasma-cleaned microscope slides by exposure to oxygen plasma for 7 s (DienerFemto plasma asher), sealing to a glass microscope slide and baking overnight at 75 $^{\circ}$C. The resultant microfluidic channels had a width of approximately 1o0 $\mu$m and a height of approximately 25 $\mu$m.

%\subsection{smFRET Measurements to Determine the Effect of Unequal Excitation on Photon Emission}
%Labelled Holliday Junctions were diluted to a concentration of 50 pM in TEN buffer (10 mM Tris, 1mM EDTA, 100 mM Nacl), pH 8.0, containing 0.01 \% Tween-20. This solution was flowed at a speed of 5 mm s$^{-1}$ through the confocal excitation volume using a microfluidic channel. Flow rate and direction was controlled using a peristaltic pump (???) The size of the confocal volume was varied between a diffraction-limited spot and a XXX mm line by varying the amplitude (pk-pk) of the ennervating triangular wave between 0 V and 5 V. Data were collected for XX minutes, using a total laser power of YY mW, corresponding to an excitation laser power (first harmonic) of 5-XXX? mW. Detected photons were binned online into time-bins of YYY $\mu$s and stored in files of 10000 bins.   

\subsection{Counting Photobleaching Steps Using TIRF Imaging}

\section{Results}
The major results of this chapter are now presented. Firstly, we show that the relationship between oligomer size and rate of photon emission shows considerable over-dispersion. We demonstrate this through comparison of simulated and real datasets from oligomers of known size. We then present our attempts to reduce excitation heterogeneity by acousto-optic modulation of the laser beam position. Finally, we present the results of a photobleaching step analsyis of the labelled Holliday junctions, which reveals that photoblinking is a major source of emision heterogeneity.

\subsection{The need for a Generative Model}
% this needs to be said more formally
To open the results section of this chapter, we present a justification for taking a model-based approach to studying aggregation. It is clear that there are many unknowns and many simplifications in our models. Here, we present some simple simulations to demonstrate the importance of understanding the processes underlying fluorescent emission. Although it is true that our models contain simplifications, it is also true that any method of analsing experimental data contains simplifications and assumptions about the structure of those data. In the case of oligomer sizing experiments, these assumptions are often not made explicit, but are based on fluorophores demontsrating poissonian emission behaviour. Here, we compare the performance of this data analysis methodology, using datasets simulated using both the poisson model of emission behaviour, and the gamma-poisson mixture model. We show that the assumptions made in the data analysis workflow do not hold when the photon emission distribution is overdispersed, leading to calculated oligomer size distributions that are both quantitatively and qualitatively incorrect. 

The largest assumption made in prior work to determine oligomer sizes using single molecule fluorescence is that there is a linear relationship between oligomer size and photon emission. In prior work~\cite{cremades2012}, oligomers were identified by using the AND criterion on time-binnned TCCD data (see Section~\ref{subect:aggregation} for details) to select time-bins that exceeded some thresholds $T_D$ and $T_A$ in the donor and acceptor channels respectively. Following event selection, the approximate size of each identified oligomer was calculated using the relationship:

\begin{equation}
\text{size} = \frac{2\cdot I_D + \gamma^{-1} I_A}{I_{\text{monomer}}}
\label{eq:size_linear}
\end{equation}

where $I_D$ and $I_A$ are the intensities (number of photons) in the donor and acceptor channels respectively, $\gamma$ is the instrumental gamma factor and $I_{\text{monomer}}$ is the mean brighness of monomer events (selected by identifying events for which $I_D \geq T_D$ but $I_A < T_A$). This means that the calculated oligomer size is simply the oserved event brightness divided by the mean event brightness for monomer events.

Fig.~\ref{fig:poisson_size_photons} shows the expected emission distributions from small oligomers of various sizes, when emission behaviour is purely poisson. This simple simulation ignores contributions from noise and assumes a mean monomer emission of 10 photons. The code used to generate these data can be found in Appendix XXX. As seen from this figure, when the emission behaviour is purely poisson, there are clear, distinct peaks in photon emission frequency for each oligomer size. As a consequence, using Eq.~\ref{eq:size_linear} to estimate the oligomer size gives approximately accurate results. We simulated oligomer emission events, sampling a uniform distribution of sizes with a monomer monomer brightness of $I_{\text{monomer}} = 10$. As we consider only single-colour excitation and emission, event selection was performed using $T_D = 10$ only and the identified fluorescent events were sized according to a simplified version of Eq.~\ref{eq:size_linear}:

\begin{equation}
\text{size} = \frac{I_D}{I_{\text{monomer}}}
\label{eq:size_linear_simple}
\end{equation}

The results, shown in Fig.~\ref{fig:poisson_size_photons_uniform} show that, apart from at the very ends of the size distribution, the calculated size distribution (blue curve) is an excellent approximation of the true size distribution (grey histogram).        

\begin{figure}
   \begin{center}
      \includegraphics*[clip=true, width=6in]{sizing/poisson_size_photons.pdf}
      \caption{Simulating oligomer photon emission using the simple poisson model. a) Simulated emission distributions from oligomers of size 1 (monomer) to 10, for simple poisson emission. b) The same distributions as in a), plotted using a logarithmic scale.}
      \label{fig:poisson_size_photons}
   \end{center}
\end{figure}


\begin{figure}
   \begin{center}
      \includegraphics*[clip=true, width=6in]{sizing/True_vs_Measured_oligomer_events_uniform_poisson_100000_hist.pdf}
      \caption{True and calculated oligomer sizes for data simulated using poisson emission behaviour. Under these conditions, the calculated oligomer size distribution (blue line) closely approximates the true size distribution (grey histogram).}
      \label{fig:poisson_size_photons_uniform}
   \end{center}
\end{figure}

However, when oligomers are simulated using the gamma-poisson mixture model, the assumption of linearity is no longer valid. As Fig.~\ref{fig:gamma_poisson_size_photons} shows, overdipsersal results in much broader distributions for which clear peaks corresponding to different oligomer sizes can no longer be discerned. As a result, estimating the oligomer size using Eq.~\ref{eq:size_linear} gives wildly inaccurate results (Fig.~\ref{fig:gamma_poisson_size_photons_uniform}).

\begin{figure}
   \begin{center}
      \includegraphics*[clip=true, width=6in]{sizing/gamma_poisson_size_photons.pdf}
      \caption{Simulating oligomer photon emission using the gamma-poisson mixure model. a) Simulated emission distributions from oligomers of size 1 (monomer) to 10, for gamma-poisson emission. b) The same distributions as in a), plotted using a logarithmic scale to show emission detail.}
      \label{fig:gamma_poisson_size_photons}
   \end{center}
\end{figure}

\begin{figure}
   \begin{center}
      \includegraphics*[clip=true, width=6in]{sizing/True_vs_Measured_oligomer_events_uniform_100000_hist.pdf}
      \caption{True and calculated oligomer sizes for data simulated using gamma-poisson emission behaviour. Under these conditions, the calculated oligomer size distribution (blue line) is an extremely poor approximation of the true size distribution (grey histogram).}
      \label{fig:gamma_poisson_size_photons_uniform}
   \end{center}
\end{figure}

This simple analysis of simulated data illustrates the importance of understanding the models that underlie a data analysis technique. If oligomers are pure poisson emitters, then Eq.~\ref{eq:size_linear} is an appropriate method to approximate the oligomer size distribution. However, if the effects of excitation heterogeneity and photobleaching cause significant over-dispersal of the emission distribution, Eq.~\ref{eq:size_linear} produces an estimation of the oligomer size distribution that is extremely inaccurate. In the following section, we compare photon emission histograms from model oligomers of known size with simulated datasets, showing the poisson distribution is a poor model of the observed photon emission distributions.   

\subsection{Understanding the Relationship Between Size and Photon Emission}
To understand the relationship between oligomer size and photon emission in a single molecule fluorescence experiment, it is necessary to have experimental datasets for which the size, or size distribution, of oligomers present is known. In a protein aggregation experiment, the aggregation reaction generates an heterogeneous solution of oligomers, but their size distribution is unknown, making them a poor model for understanding photon emission behaviour. 

To overcome this problem, we use a labelled DNA Holliday Junction as a model oligomer. We prepared Holliday Junctions labelled with between one and four fluorophores (Alexa Fluor 488), enabling us to collect data from solutions of pure monomer, dimer, trimer and tetramer models. We compare experimental data collected using these oligomer models from datasets simulated using either the pure poisson or the gamma-poisson emission models. The results are very informative.

Fig.~\ref{fig:HJ_oligomers} displays histograms of the full photon frequency distribution for Holliday Junctions labelled with between one and four fluorophores. These data are displayed on a logarithmic scale, so that key features of the emission distribution can be identified. Several points of interest can be noted. Firstly, similarly to the dual-labelled monomeric species described in the previous chapter, no clear distinction is observed between the poissonian noise distribution and the distribution of photons from fluorescent events. This is true even when multiple fluorophores are present. Consequently, we present the entire dataset, including time-bins containing only noise photons, to avoid distorting the observed brightness distributions. Secondly, no clear peaks -- corresponding to oligomers of different sizes -- are observed. Instead, there is extensive overlap between the photon emission distributions for oligomers of different sizes, with the main differentiator being an increase in the length of the tail for species carrying more flurophores. Finally, it can also be observed that at the opposite end of the distribution, the frequency of events also increases for multimeric species: the monomer model has the highest number of bins containing zero photons. For all other photon counts, multimeric species show an increased frequency of observation. 

This is in contrast to datasets simulated using the pure poisson emission model (Fig.~\ref{fig:poisson_oligomers}). In these datasets, separate peaks corresponding to oligomers of different sizes can be clearly distinguished. Furthermore, rather than a monotonic decrease in frequency with number of observed photons, the poisson model shows an initial decrease in event frequency at low numbers of photons, corresponding to the noise distribtuion, followed by a second peak, corresponding to oligomer events. This produces a clear separation between the noise and event emission distributions, which would allow a simple threshold to select oligomer events. 

On the other hand, the Holliday Junction photon distributions show better resemblance to datasets simulated using the gamma-poisson mixture model (Fig.~\ref{fig:gamma_poisson_oligomers}). Like the Holliday Junction data, these datasets show a monotonic decrease in event frequency with increasing photons and display longer tails of rare, bright events when more fluorophores are present Furthermore, these datasets do not distinguish peaks corresponding to oligomers of different sizes. 

This comparison raises two important points. Firstly, the emission behaviour of our oligomeric models in a single molecule fluorescence experiment is extremely poorly approximated by the poisson model for which an assumption of linear increase in photon emission with oligomer size would be valid. It is therefore clearly inappropriate to attempt using a simple thresholding method to calculate oligomer size distributions in aggregation reactions. These are likely to lead to the wildly inaccurate estimations of oligomer size distributions shown in Fig.~\ref{fig:gamma_poisson_size_photons_uniform}.

Secondly, as the gamma-poisson mixture model appears to better approximate the emission distribution actually observered, it is possible that inferring the parameters of this model conditioned on an oligomeric dataset would enable accurate determination of the emission paramaters of at least oligomers of a single size. The next section describes our attempt to infer the parameters of the single-colour model, conditioned on datasets from the synthetic Holliday Junctions.  

\begin{figure}
   \begin{center}
      \includegraphics*[clip=true, width=6in]{sizing/Holliday_Junctions.pdf}
      \caption{Histograms of experimental photon emission distributions for labelled Holliday Junctions. All time-bins are shown. Data from monomer, dimer, trimer and tetramer models are show in dark blue, green, red and cyan respectively.}
      \label{fig:HJ_oligomers}
   \end{center}
\end{figure}

\begin{figure}
   \begin{center}
      \includegraphics*[clip=true, width=6in]{sizing/poisson_oligomers_log.pdf}
      \caption{Histograms of photon emission distributions for model oligomers simulated using a pure poisson model. Data from monomer, dimer, trimer and tetramer models are show in dark blue, green, red and cyan respectively.}
      \label{fig:poisson_oligomers}
   \end{center}
\end{figure}


\begin{figure}
   \begin{center}
      \includegraphics*[clip=true, width=6in]{sizing/gamma_poisson_oligomers_log.pdf}
      \caption{Histograms of photon emission distributions for model oligomers simulated using a gamma-poisson mixture model. Data from monomer, dimer, trimer and tetramer models are show in dark blue, green, red and cyan respectively.}
      \label{fig:gamma_poisson_oligomers}
   \end{center}
\end{figure}

\clearpage

\subsection{Inferring Event Brightness Using the Gamma-Poisson Model}
The previous section described comparing experimental data from model oligomers of known sizes with simulated datasets. We showed that poisson emission is an extremely poor model of the behaviour of fluorescently labelled oligomers in a single molecule fluorescence experiment. In this section, we use our Metropolis-Hasting sampler, described in Chapter~\autoref{chap:inference} to infer the parameters of our gamma-poisson mixture model, conditioned on single molecule datasets. We first present parameter inference on datasets simulated using this model, demonstrating effective performance. We then present the results of parameter inference conditioned on experimental datasets, showing that our lack of knowledge of key model parameters means that the model is underspecified, making accurate parameter inference difficult.

We simulated four datasets, consisting of oligomers of known size (from monomers to tetramers) using the parameters shown in Table~\ref{tab:oligomer_params}. Then, we used the Metropolis Hastings sampler to infer the mean oligomer and noise emission rates, as well as the oligomer concentration. The initial values used are shown in Table~\ref{tab:initial_values}; the inferred values are shown in Fig.~\ref{fig:gamma_poisson_oligomers}. Two example fits are shown in Fig.~\ref{fig:hist_fit_trimer}

\begin{center}
\begin{table}[!ht]
\caption{Parameters used in simulating emission datasets for oligomers of known size.}
%\begin{adjustwidth}{-2.25in}{0in} % Comment out/remove adjustwidth environment if table fits in text column.
\begin{tabular}{|l|l|}
\hline
{\bf Parameter} & {\bf Value}\\ \hline
$\lambda_{\text{prot}}$ & 0.05\\
$\lambda_{\text{NB}}$ & 1.0\\
$\lambda_{\text{DB}}$ (monomer) & 10.0\\
$\lambda_{\text{DB}}$ (dimer) & 20.0\\
$\lambda_{\text{DB}}$ (trimer) & 30.0\\
$\lambda_{\text{DB}}$ (tetramer) & 40.0\\
$R_{\text{blue}}$ & 1.0\\
$\gamma_{\text{ins}}$ & 1.0\\
$n_{\text{samples}}$ & 10000\\ \hline
\end{tabular}
\label{tab:oligomer_params}
%\end{adjustwidth}
\end{table}
\end{center}

\begin{center}
\begin{table}[!ht]
\caption{Initial values used in inferring  oligomers of known size.}
%\begin{adjustwidth}{-2.25in}{0in} % Comment out/remove adjustwidth environment if table fits in text column.
\begin{tabular}{|l|l|}
\hline
{\bf Parameter} & {\bf Value}\\ \hline
$\lambda_{\text{prot}}$ & 0.1\\
$\lambda_{\text{NB}}$ & 1.0\\
$\lambda_{\text{DB}}$ & 20.0\\
\hline
\end{tabular}
\label{tab:initial_values}
%\end{adjustwidth}
\end{table}
\end{center}

\begin{figure}
   \begin{center}
      \includegraphics*[clip=true, width=6in]{sizing/divergence.pdf}
      \caption{Plot of inferred vs true values of $\lambda_{\text{DB}}$. The dashed line shows the expected trend, but the inferred values are underestimated for high values of $\lambda_{\text{DB}}$. Error bars show the standard deviation of 100 samples.}
      \label{fig:gamma_poisson_oligomers}
   \end{center}
\end{figure}

\begin{figure}
   \begin{center}
      \includegraphics*[clip=true, width=6in]{sizing/oligomer_marginals.pdf}
      \caption{Histograms of photon emission distributions for simulated a) monomers and b) trimers. The overlaid blue circles show the  number of photons predicted by the gamma-poisson mixture model, using parameters inferred from the dataset using the Metropolis sampler.}
      \label{fig:hist_fit_trimer}
   \end{center}
\end{figure}

The results are mixed. Although inference performs well for small oligomers (monomers and dimers), when the mean event brightness is very high (pure trimer and pure tetramer simulations), the true value of $\lambda_{\text{DB}}$ is underestimated. Despite the considerable error, this still produces expected emission frequencies that are a good approximation of the true distribution (compare the histogram and the overlaid blue circles in Fig.~\ref{fig:hist_fit_trimer} B). The likely cause of this estimation error is the overdispersal. In these simulations, the value of $R_{\text{blue}}$, the overdispersal parameter was 1.0. This produces an extremely broad distribution: in particular, for high mean values $\lambda_{\text{DB}}$, the distribution has a very long tail and the modal observed value is far from the mean of the distribution. As a consequence of this, the possible range of parameters values that provide a good fit to the observed data is very wide and the stationary distribution is broad and flat, allowing many values of $\lambda_{\text{DB}}$ to provide a good approximation of the dataset.

It should be noted that this poor performance is not caused by the sampler not converging. As can be seen from the tight error bars in Fig.~\ref{fig:gamma_poisson_oligomers}, the accepted samples are drawn from a small range of values. Furthermore, the initial conditions do not affect the final value reached by the sampler: although a longer burn-in phase is required when the sampler is initialised using values far from the true parameter values, sampling converges to the same (incorrect) values (Fig.~\ref{fig:convergence}).

This performance is disappointing. The datasets we have simulated are a greatly simplified approximation of a true single molecule fluorescence dataset from a protein aggregation experiment: they contain ``oligomers" of only a single size; many sources of error, such as cross-talk, are entirely neglected and all model parameters are known. Despite knowing that the model for which we are inferring parameters is precisely the model that generated our datasets, we are unable to recover the correct parameter values. This suggests that our model-based approach will struggle similarly to extract meaningful parameters from experimental datasets. Our attempt to do so is summarised in the next section.

\begin{figure}
   \begin{center}
      \includegraphics*[clip=true, width=6in]{sizing/Convergence.pdf}
      \caption{Convergence of the inferred value of $\lambda_{\text{DB}}$ during burn-in for simulations of the trimeric oligomer. The black dashed line represents the true value of $\lambda_{\text{DB}}$.}
      \label{fig:convergence}
   \end{center}
\end{figure}

\subsection{How Bright Are Holliday Junction Events}
This section discusses our attempt to infer parameters of the gamma-poisson mixture model conditioned on data from our Holliday Junction model oligomers. Firstly, we provide a detailed comparison of the simulated and experimental datasets, identifying probable errors in the model. Secondly, we compare different methods of estimating the event brightness of fluorescence emission from the Holliday Junction oligomers. Finally, we consider experimental techniques to understand sources of discrepancy between our model and the observed data.

Fig.~\ref{fig:HJ_sim_overlay} A and B show emission distribution histograms for the monomer and trimer model oligomers, overlaid by simulated monomer and trimer emission distributions, simulated using the gamma-poisson mixture model. Although there is rough agreement between the two distributions, several discrepancies can be noted. Firstly, as is particularly evident in the comparison of monomer data (Fig.~\ref{fig:HJ_sim_overlay} A), the experimental distribution shows a much longer tail of bright events. Secondly, the experimental datasets display fewer events of intermediate fluorescence. These discrepancies suggest that the gamma-poisson mixture model, as parameterised in our simulations, is not a perfect model for fluorescent emission.

\begin{figure}
   \begin{center}
      \includegraphics*[clip=true, width=6in]{sizing/HJ_overlays.pdf}
      \caption{Overlays of simulated and experimental photon emission distributions for a) monomer and c) trimer oligomers. The blue histograms display experimental data from the Holliday Junction model oligomers. The green histograms display data from simulations using the gamma-poisson mixture model.}
      \label{fig:HJ_sim_overlay}
   \end{center}
\end{figure}

The most likely source of this discrepancy is incorrect parameterisation of the gamma-poisson mixture distribution. This occurs because the true values of several parameters of this distribution are unknown. In particular, in our simulations, we assume that the overdispersal parameter, $R_{\text{blue}}$, is 1.0. However, for expermimental data, this parameter is unknown. By comparison of datasets simulated using different values of $R_{\text{blue}}$ with experimental data, it is possible to estimate the experimental value of $R_{\text{blue}}$ (Fig.~\ref{fig:R_blue}). Unfortunately, however, the parameter inference is extremely sensitive to the value of $R_{\text{blue}}$ (see Fig.~\ref{fig:vary_Rblue} and Table~\ref{tab:R_blue}), meaning that without a good estimate of this value, it is very difficult to infer physically meaningful values.

To overcome this issue, we attempted to co-infer $R_{\text{blue}}$ along with other model parameters, by allowing it to become a variable rather than a constant in our model. Unfortunately, this was also unsuccessful, as the model converges to extremely low values of $R_{\text{blue}}$, for which the inferred oligomer concentration is concommitantly unphysically high and the mean event brightness very low (Table~\ref{tab:R_blue}). 


\begin{figure}
   \begin{center}
      \includegraphics*[clip=true, width=6in]{sizing/varying_R_blue.pdf}
      \caption{Photon emission distributions for datasets simulated with different values of the overdispersion parameter $R_{\text{blue}}$. The values used are shown in the legend. For high values of $R_{\text{blue}}$, the degree of overdispersal is small, and the fluorescence emission distribution resembles a pure poisson distribuion (black histogram). For small values of $R_{\text{blue}}$, the amount of overdispersal is much greater and the tail of the distribution is lengthened (green histogram).} 
      \label{fig:R_blue}
   \end{center}
\end{figure}

\begin{figure}
   \begin{center}
      \includegraphics*[clip=true, width=6in]{sizing/vary_Rblue.pdf}
      \caption{Fitting a dataset from the monomer model Holliday Junction, using different fixed values of $R_{\text{blue}}$: a) $R_{\text{blue}}$ = 0.1, b) $R_{\text{blue}}$ = 0.5. c) $R_{\text{blue}}$ = 1.0, d) $R_{\text{blue}}$ = 2.0. Although the data superficially seems better fitted when $R_{\text{blue}}$ is smaller, low values of $R_{\text{blue}}$ overfit random variations and produce unphysical parameter values (see Table~\ref{tab:R_blue}).} 
      \label{fig:vary_Rblue}
   \end{center}
\end{figure}

\begin{center}
\begin{table}
\caption{Inferred parameter values conditioned on a dataset from the monomer model Holliday Junction, using different fixed values of $R_{\text{blue}}$. At low values of $R_{\text{blue}}$, overfiting occurs and the inferred parameters are unphysical. When we try to infer the value of $R_{\text{blue}}$ (bottom row), the data is also overfitted.}
%\begin{adjustwidth}{-2.25in}{0in} % Comment out/remove adjustwidth environment if table fits in text column.
\begin{tabular}{|l|l|l|l|}
\hline
{\bf $R_{\text{blue}}$} & {\bf $\lambda_{\text{noise}}$} & {\bf $\lambda_{\text{prot}}$} & {\bf $\lambda_{\text{DB}}$}\\ \hline
0.1 & 0.44 & 0.215 & 2.769 \\
0.5 & 0.46 & 0.059 & 9.784 \\
1.0 & 0.47 & 0.042 & 13.156 \\
2.0 & 0.48	& 0.035	& 15.193 \\
inferred = 0.029 & 0.44 & 1.006 & 0.591\\ \hline
\end{tabular}
\label{tab:R_blue}
%\end{adjustwidth}
\end{table}
\end{center}

This poor behaviour is caused by overfitting. There is insufficient information in the datasets to distinguish between a solution with manny molecules each of which emit only a few photons and a model in which few molecules emit many photons in each rare excitation event. Moreover, a model in which there are multiple dim emitters in the confocal volume at any one time gives greater flexibility to exactly fit small, variations in the observed emission distributions. As a result, dim emitters are preferred as they allow the model to more closely approximate this random noise, resulting in clearly incorrect  and unphysical fits.

This has considerable implications for our ability to fit oligomer data using the Metropolis sampler, as the results of sampling are extremely dependent on the value of the overdispersion parameter $R_{\text{blue}}$, which we are not able to infer accurately. Consequently, extension of the inference tool to oligomer sizing is unsuccessful. There is insufficient information obtained from an experiment to accurately calculate the mean event emission brightness or to separate the noise and fluorescence emission distributions, even for molecules carrying a known, uniform number of fluorophores. However, as we have shown (Fig.~\ref{fig:gamma_poisson_size_photons_uniform}), the degree of overdispersal means that simple thresholding analyses are also entirely inappropriate to accurately calculate oligomer size distributions. To obtain accurate size information from a single molecule fluorescence experiment, it is necessary to reduce the sources of heterogeneity, so that photon emission can be well approximated using a pure poisson distribution (Fig.~\ref{fig:poisson_size_photons_uniform}). The following section describes our attempts to understand sources of emission overdispersal. The chapter concludes with a discussion of possible methods to reduce this heterogeneity.   

%\subsection{Attempting to Remove Sources of Heterogeneity}
\subsection{Photobleaching Steps Analysis Reveals Additional Source of Overdispersal}
This section describes the results of a photobleaching step analysis of the Holliday Junction model oligomers. We inially performed these experiments to verify the presence of multiple fluorophores on the dimer, trimer and tetramer constructs; however it also revealed that photoblinking is likely to be a major source of emission heterogeneity in single molecule fluorescence experiments.

The final results of these experiments, summarized in Fig.~\ref{fig:photobleaching}, show that, although we do observe an increase in the mean number of photobleaching steps for Holliday Junctions labelled with more fluorophores, there is considerable variation in the number of photobleaching steps seen. This could imply that the Holliday Junctions are badly formed and therefore not a homogeneous sample. However, this result should be considered in conjunction with Table~\ref{tab:photobleaching}, which summarizes the event selection and fitting steps of the photobleaching analysis. As shown, of all loci identified as emitters, over XXX \% were discarded as being unsuitable for photobleaching step counting, because they contained unclear transitions. A typical example of a trace that could not be used is shown in Fig.~\ref{fig:blinking}. Both increases and decreases in the observed photon emission are clearly observed. This behaviour is caused by photoblinking~\cite{blinking}, reversible transitions of the fluorophores between bright and dim or dark emission states. A video of one photobleaching experiment, showing the extent of this blinking behaviour, can be see in Appendix~\ref{A1_blinking_video}.

The extensive photoblinking behaviour that we observe here in Alexa Fluor 488 is cause for concern. This blinking occurs on a timescale comparable with dwell-time in the confocal volume during a diffusion-based single molecule experiment. Experimental techniques that aim to reduce the excitation field heterogeneity will not affect heterogeneity caused by this photo-physical effect. Consequently, any attempt to accurately size individual oligomeric molecules, or to characterize a size distribution based on the observed fluorescence emission brightness distribution must take this source of heterogeneity into account.      

\section{Conclusions}
\subsection{Complex Relationship between Size and Photon Emission}
This chapter described a model-based approach to understanding the size distribution of oligomers in a single molecule fluorescence study of protein aggregation. Through a comparison of experimental datasets with simple simulations, we were able to show that the photon emission distribution observed, even from a simple, homogenous solution of labelled molecules, is much better modeled by an overdispersed poisson distribution (gamma-poisson mixture) that a simple poisson distribution.

We showed, using simulated datasets, that this overdispersion makes the number of observed photons and extremely poor estimator for molecular size (Fig.~\ref{fig:gamma_poisson_size_photons_uniform}), resulting in a huge divergence between true and calculated size distributions. Furthermore, because we were not able to accurately estimate or measure the value of the overdispersion parameter $R_{\text{blue}}$, we were unable to use the Metropolis sampler described in the previous chapter to infer reliable estimates of our model parameters. Consequently, we conclude that unless stragegies can be found for reducing photon emission heterogeneity, it is extremely difficult to accurately infer oligomer sizes based on their emission behaviour in single molecule fluorescence experiments.

\subsection{Implications for Future Work on Molecular Sizing}
Reducing emission hetereogenity in order to accurately infer oligomer sizes requires a dual approach. Firstly, it is necessary to suppress photoblinking behaviour, so that reversible transitions between light and dark states, which would broaden the emission distribution, do not occur. Certain fluorophores, such as Rhodamine 6G, are known to show extensive photoblinking behaviour, and a heterogeneous distribution of dark state lifetimes~\cite{zondervan03}. Alexa Fluor 488, the fluorophore used in our study of Holliday Junctions, has extensive structural similarity to Rhodamine G6, with both dyes having a heterocyclic aromatic structure, derived from fluorone (Fig.~\ref{fig:fluorone}). Consequently, it is likely that the photoblinking behaviour observed in Alexa 488 is caused by a single electron transfer allowing population of a radical state, similar to that observed in Rhodamine~\cite{zondervan03}. 


\begin{figure}
   \begin{center}
      \includegraphics*[width=4in]{sizing/fluorone.pdf}
      \caption{The chemical structures of A) the para isomer of Alexa Fluor 488 and B) Rhodamine 6G are very similar. Both are derived from the flurone ring system shown in C).}
      \label{fig:fluorone}
   \end{center}
\end{figure}

Several systems have been derived that reduce the occurrence of photoblinking in organic fluorophores. These systems, including the glucose oxidase - catalase (GODCAT) system~\cite{joo2006}, Trolox-based buffers~\cite{rasnik2006} and protocatechuic acid (PCA)/protocatechuate-3,4-dioxygenase (PCD)~\cite{aitken2008}, typically scavenge triplet oxygen from the solution, removing the main source of electrons for the dark state entry reaction. Although our buffers were freshly degassed, to reduce the concentration of dissolved oxygen, we did not use an oxygen scavenging system. We suggest that use of such a system could help to reduce the extensive photoblinking that prevented oligomer quantification. 

Secondly, emission heterogeneity can also be reduced by creating a more uniform exciation within the confocal detection volume. The gaussian shape of the exciting laser beam means that the pathway taken by molecules diffusing through the confocal volume affects both the intensity and duration of excitation, broadening the observed emission distribution. Several recent studies identify methods for reducing the confocal excitation heterogeneity. Liu and Wang~\cite{liu2008} use a cylindrical lens to create a uniform excitation field in one dimension; they then use a square pinhole combined with nanofluidic channels to detect only molecules that pass precisely through this uniform excitation field. A similar approach is taken by Tyagi and coworkers~\cite{tyagi2014}, who use reversible flattening of microfluidic channels to observe emission from fluorescent molecules confined in the TIRF field. In this work, the channels were also perfused with nitrogen gas, enabling removal of oxygen molecules capable of inducing blinking. We suggest that adopting such techniques, which homogenize both excitation duration and intensity, could be used to reduce emission heterogeneity and hence enable accurate quantification of oligomer sizes based on fluorescence emission behaviour.

Finally therefore, we would like to conclude this chapter on an optimistic note. Although we were unsuccessful in quantifying oligomer sizes, our model based approach has enabled us to identify limitations in current experimental practices that are currently preventing single molecule fluorescence approaches from accurately determining oligomer sizes. Based on these limitations, we are able to suggest several possible modifications to the experimental protocol that could improve the accuracy of this technique. We hope that these suggestions prove helpful to future researchers working in this field. 